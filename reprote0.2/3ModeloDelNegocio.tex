%=========================================================
\chapter{Modelo del Negocio}	
\label{cap:reqSist}

	En este capítulo se modela la {\em Arquitectura del negocio} la cual está conformada por los ({\em Términos} y {\em Hechos del negocio}), Arquitectura de procesos y las {\em Reglas del negocio}. Primero se especifica brevemente el {\em Contexto} en el que los términos tienen significado.
	
%----------------------------------------------------------
\section{Contexto}

%	\cdtInstrucciones{El contexto debe explicar bajo que ambiente los términos del negocio son aplicables y proporcionar información general para su comprensión inicial.\\}
	Consorcio Deportivo S.A. de C.V.  es una empresa nueva, que busca crecer a nivel nacional, para esto se unió con otras cadenas deportivas. Al poner en funcionamiento sus gimnasios en Monterrey, Guadalajara y en la Ciudad de México, se dieron cuenta que las demás cadenas tienen procesos, procedimientos e información diferentes; algunas personas que viajan y que quieren seguir con su rutina de ejercicio, se les negó el ingreso a las instalaciones ya que la información no se encuentra unificada en todas las sucursales.
	
	Las actividades que pueden ser contratadas tienen problemas ya que no tienen control alguno. Estas actividades son: área de piscina, área de instrumentos (pesas, caminadoras, etc.), área de canchas (fútbol, tennis, etc.).  De igual manera las membresías carecen de control alguno, ya que los miembros no ponen atención a la fecha de renovación de estas. Los miembros a los que se les notificó que tenían su pago de renovación atrasado se les negaba su ingreso a lo que el miembro se respalda tras la escusa que ya realizó el pago, ya que no hay manera de comprobar de manera inmediata la veracidad de esto, se le da acceso al cliente, esto se traduce en pérdidas.
	
%---------------------------------------------------------
\section{Términos del Negocio}
\label{sec:terminosDeNegocio}

\begin{description}
	% Ejemplo de un término literal.
	%\item[\hypertarget{tAutomovil}{Automóvil:}] ({\em es un tipo de \hyperlink{tVehiculo}{Vehículo}}) De cuatro ruedas con capacidad de 5 a 9 personas. 
	% Ejemplo de un término de entidad
	%\item[\hypertarget{tCliente}{Cliente:}] Se refiere a todas las personas físicas y morales que \hyperlink{tRenta}{rentan} o han rentado un \hyperlink{tVehiculo}{vehículo}.
	
	\item[\hypertarget{tJefe}{Jefe de Inmobiliario:}] ({\em es un tipo de \hyperlink{tEmpleado}{Empleado}}) Es el empleado que tiene mayor rango de todos y no tiene superior, a diferencia de los demás.	
	\item[\hypertarget{tEmpleado}{Empleado:}] Se refiere a cualquier persona que labore en la empresa.
	
	\item[\hypertarget{tGerente}{Gerente de Inmobiliario:}] ({\em es un tipo de \hyperlink{tEmpleado}{Empleado}}) Es el empleado que tiene mayor rango dentro de la sucursal en la que trabaja.	
	\item[\hypertarget{tEmpleado}{Empleado:}] Se refiere a cualquier persona que labore en la empresa.
	 
	\item[\hypertarget{tcliente}{Cliente:}] {\em Es el aquel que adquirió una membresía}.
	
	\item[\hypertarget{tInvitado}{Invitado:}] {\em es un tipo de usuario que tiene adquirido algun servicio con el consorcio, pero no tiene una membresía adquirida.} 

	
	
%	\brTermSensor{tVelocimetro}{Velocímetro:}{Velocidad de un Vehículo.}{Kilometros/hora.}{Constantemente siempre que el \cdtRef{tVehiculo}{vehículo} esté encendido.}
\end{description}

%----------------------------------------------------------
\section{Modelo del dominio del problema}
\label{sec:hechosDeNegocio}


%- - - - - - - - - - - - - - - - - - - - - - - - - - - - - 
\subsection{Modelo del dominio del problema}

	El modelo del dominio del problema se muestra en la figura~\ref{fig:modeloDeDominio}, a continuación se describen cada una de las entidades y sus relaciones.
	
\begin{figure}[htbp!]
	\begin{center}
		\includegraphics[angle=90,width=.95\textwidth]{images/BD/bd1}
		\caption{Modelo del dominio del problema parte 1}
		\label{fig:modeloDeDominio}
	\end{center}
\end{figure}

\begin{figure}[htbp!]
	\begin{center}
		\includegraphics[angle=90,width=.5\textwidth]{images/BD/bd2}
		\caption{Modelo del dominio del problema parte 2}
		\label{fig:modeloDeDominio}
	\end{center}
\end{figure}

\begin{figure}[htbp!]
	\begin{center}
		\includegraphics[angle=90,width=.95\textwidth]{images/BD/bd3}
		\caption{Modelo del dominio del problema parte 3}
		
	\end{center}
\end{figure}

\begin{figure}[htbp!]
	\begin{center}
		\includegraphics[angle=90,width=.7\textwidth]{images/BD/bd4}
		\caption{Modelo del dominio del problema parte 4}
		
	\end{center}
\end{figure}
%- - - - - - - - - - - - - - - - - - - - - - - - - - - - - 

\newenvironment{cdtEntidad}[2]{%
	\def \varBusinessEntityId{#2}%
	\hypertarget{#1}{\hspace{1pt}}%
	\newline%
	\noindent{\includegraphics[width=\textwidth]{images/uc/classRule}}%
	\vspace{-25pt}%
	\subsection{Entidad: #2}%
	\noindent\begin{longtable}{|p{.2\textwidth}| p{.15\textwidth} | p{.46\textwidth} |p{.08\textwidth} |}%
		\hline%
		\multicolumn{4}{|c|}{{\cellcolor{colorSecundario}\color{white}Atributos}}\\ \hline%
		{\cellcolor{colorAgua}Nombre} &%
		{\cellcolor{colorAgua}Tipo} &%
		{\cellcolor{colorAgua}Descripción} &%
		{\cellcolor{colorAgua}Requerido}%
		\\ \hline%
		\endhead%
	}{%
	\end{longtable}%
}

\newcommand{\brAttr}[5]{%
	{\bf\hypertarget{\varBusinessEntityId:#1}{#2}} & {\em{#3}} & {#4} & #5 \\\hline
}

\newcommand{\cdtEntityRelSection}{%
	\multicolumn{4}{|c|}{{\cellcolor{colorSecundario}\color{white}Relaciones}}\\ \hline%
	{\cellcolor{colorAgua}Tipo de relación} &%
	{\cellcolor{colorAgua}Entidad} &%
	\multicolumn{2}{|c|}{{\cellcolor{colorAgua}Rol}}
	\\ \hline%
}

\newcommand{\brRelComposition}{{\color{colorPrincipal}$\Diamondblack$\hspace{-1pt}---Composición}}
\newcommand{\brRelAgregation}{{\color{colorPrincipal}$\Diamond$\hspace{-1pt}---Agregación}}
\newcommand{\brRelGeneralization}{{\color{colorPrincipal}$\lhd$\hspace{-1pt}---Generalización}}

\newcommand{\brRel}[3]{%
	{\em{#1}} & {\bf{#2}} & \multicolumn{2}{|l|}{#3}\\\hline
}

%- - - - - - - - - - - - - - - - - - - - - - - - - - - - - 

\begin{cdtEntidad}{Cliente}{Cliente}
	\brAttr{Cliente}{Cliente}{Id}{Número de registro para identificar al cliente.}{Sí}
	\brAttr{nombre}{Nombre}{Palabra larga}
	{Nombrecompleto del cliente.}{Sí}
	\brAttr{Domicilio}{Domicilio}{Palabra }
	{Domicilio del cliente.}{Sí}
	\brAttr{Correo}{Correo}{Correo}{Correo del cliente para envio de circulares o avisos.}{No}
	\brAttr{Telefono}{Telefono}{Telefono}{Telefono de contacto del cliente}{No}
	\brAttr{NumeroEmergencia}{Numero de emergencia}{Telefono}
	{Telefono de contcto en caso de alguna emergencia.}{Sí}
	\brAttr{nacimiento}{Nacimiento}{Fecha}
	{Fecha de nacimiento del cliente.}{Sí}
	\brAttr{Estatura}{Estatura}{Flotante}{Estatura del cliente}{NO}
	\brAttr{Peso}{Peso}{Flotante}
	{Peso del cliente.}{NO}
	\brAttr{Alergias}{Alergias}{Booleano}
	{Si el cliente cuenta con alguna alergia.}{Sí}	
	\brAttr{Enfermedades}{Enfermedades}{Booleano}
	{Si el cliente cuenta con alguna enfermedad crónica.}{SI}
	\brAttr{Medicación}{Medicación}{Booleano}
	{Si el cliente cuenta con medicación.}{SI}
	\brAttr{sucursal}{sucursal}{ID}{Numero de registro para identificar la sucursal a la que pertenece}{SI}
	\brAttr{ID_Membresia}{Membresía}{ID}
	{Numero de registro para identificar la membresia qye tiene el cliente.}{SI}
	\brAttr{Nombre}{Usuario}{Nombre de y usuario}{Nombre de usuario para iniciar sesion}{SI}
	
	\cdtEntityRelSection
	\brRel{\brRelGeneralization}{Empleado}{Un \hyperlink{Gerente}{Gerente} es un \hyperlink{Empleado}{Empleado}}	
\end{cdtEntidad}
%- - - - - - - - - - - - - - - - - - - - - - - - - - - - - 

\begin{cdtEntidad}{Gerente}{Gerente}
	\brAttr{Empleado}{Empleado}{Id}{Número de registro para identificar al empleado.}{Sí}
	\brAttr{Antigedad}{Antigedd}{Entero Corto}
	{Antigedad en la Sucursal.}{Sí}
	\brAttr{Domicilio}{Domicilio}{Palabra }
	{Domicilio del cliente.}{Sí}
	
	
	\cdtEntityRelSection
	\brRel{\brRelComposition}{Sucursal}{Un \hyperlink{Cliente}{Cliente} pertenece a una \hyperlink{Sucursal}{Sucursal}}	
	\brRel{\brRelAgregation}{Membresía}{Un \hyperlink{Cliente}{Cliente} tiene una \hyperlink{Membresia}{Membresia}}	
\end{cdtEntidad}
%- - - - - - - - - - - - - - - - - - - - - - - - - - - - - 

\begin{cdtEntidad}{Instructor}{Instructor}
	\brAttr{Empleado}{ID}{Id}{Número de registro para identificar al empleado.}{Sí}
	\brAttr{Nivel}{Nombre}{entero corto}
	{Nivel en el que se encuentra en la sucurusal.}{Sí}
	
	
	\cdtEntityRelSection
	\brRel{\brRelGeneralization}{Empleado}{Un \hyperlink{Instructor}{Instructor} es un \hyperlink{Empleado}{Empleado}}	
\end{cdtEntidad}


%- - - - - - - - - - - - - - - - - - - - - - - - - - - - - 

\begin{cdtEntidad}{Recepcionista}{Recepcionista}
	\brAttr{Empleado}{ID}{Id}{Número de registro para identificar al empleado.}{Sí}
	\brAttr{Especialidad}{Especialidad}{Nombre corto}
	{Especialidad de la recepcionista en alguna área.}{Sí}
	
	
	\cdtEntityRelSection
	\brRel{\brRelGeneralization}{Empleado}{Un \hyperlink{Recepcionista}{Recepcionista} es un \hyperlink{Empleado}{Empleado}}	
\end{cdtEntidad}

%- - - - - - - - - - - - - - - - - - - - - - - - - - - - -

\begin{cdtEntidad}{Vendedor}{Vendedor}
	\brAttr{Empleado}{ID}{Id}{Número de registro para identificar al empleado.}{Sí}
	\brAttr{Membresias}{Venta}{entero corto}
	{Numero de membresias vendidas desde su ingreso.}{NO}
	
	
	\cdtEntityRelSection
	\brRel{\brRelGeneralization}{Empleado}{Un \hyperlink{Recepcionista}{Recepcionista} es un \hyperlink{Empleado}{Empleado}}	
\end{cdtEntidad}

%- - - - - - - - - - - - - - - - - - - - - - - - - - - - -

\begin{cdtEntidad}{Sucursal}{Sucursal}
	\brAttr{Sucursal}{Sucursal}{Id}{Número de registro para identificar una sucursal.}{Sí}
	\brAttr{nombre}{Nombre}{Palabra corta}
	{Nombrecompleto de la sucursal.}{Sí}
	\brAttr{Domicilio}{Domicilio}{Palabra larga}
	{Domicilio de la sucursal.}{Sí}
	\brAttr{Telefono}{Telefono}{Telefono}{Telefono de contacto para la sucursal}{SI}
	\brAttr{Hora}{Apertura}{hora}{Hora en la que empieza a dar servicio la sucursal}{SI}
	\brAttr{Hora}{Cierre}{hora}
	{hora en la que se termina el sevicio en la sucursal.}{SI}
	
\end{cdtEntidad}
%- - - - - - - - - - - - - - - - - - - - - - - - - - - - - 

\begin{cdtEntidad}{Venta}{Venta}
	\brAttr{ID}{Venta}{Id}{Número de registro para identificar la venta.}{Sí}
	\brAttr{Fecha}{Fecha}{fecha}
	{Fecha en la que se realizo la venta.}{Sí}
	\brAttr{Monto}{Monto}{Flotante }
	{Monto de la venta.}{Sí}
	\brAttr{ID}{Empleado}{ID}{Numero de registro para identificar el empleado que vendio la membresia}{si}
	
	
	\cdtEntityRelSection
	\brRel{\brRelComposition}{Vendedor}{Una \hyperlink{Venta}{Venta} tiene un \hyperlink{Vendedor}{Vendedor} que realizó la venta}	
	\brRel{\brRelComposition}{Membresía}{Una \hyperlink{Venta}{Venta} tiene una \hyperlink{Membresia}{Membresia} asignada}
	\brRel{\brRelComposition}{Cliente}{Ua \hyperlink{Venta}{Venta} tiene un \hyperlink{Cliente}{Cliente} que la solicitó}	
\end{cdtEntidad}
%---------------------------------------------------------

	\begin{cdtEntidad}{Actividad}{Actividad}
	\brAttr{ID}{Actividad}{Id}{Número de registro para identificar la Actividad.}{Sí}
	\brAttr{Nombre}{Nombre}{Nombre corto}
	{Nombre de la actividad.}{Sí}
	\brAttr{Descripcion}{Descripcion}{Nombre largo}
	{Descripcion de la actividad.}{Sí}
	\brAttr{Fecha}{Creacion}{fecha}{Fecha en la que se creo la Actividad}{SI}
	\brAttr{Cupo}{Cupo}{Entero}{Cupo máximo de la actividad}{SI}
	\brAttr{ID}{Area}{ID}{Numero de registro para identificar el area a la que pertenece la actividad}{SI}
	\brAttr{ID}{Actividad}{ID}{Numero de registro para identificar el tipo de actividad}{SI}
	
	\cdtEntityRelSection
	\brRel{\brRelComposition}{Area}{Una \hyperlink{Actividad}{Actividad} tiene un \hyperlink{Area}{Area} asignada}	
	\brRel{\brRelComposition}{TipoActividad}{Una \hyperlink{Actividad}{Actividad} tiene un \hyperlink{TipoActividad}{TipoActividad}}	
\end{cdtEntidad}

%---------------------------------------------------------

\begin{cdtEntidad}{Area}{Area}
	\brAttr{ID}{Area}{Id}{Número de registro para identificar un area.}{Sí}
	\brAttr{Nombre}{Nombre}{Nombre corto}
	{Nombre del area.}{Sí}
	\brAttr{Descripcion}{Descripcion}{Nombre largo}
	{Descripcion del area.}{Sí}
	\brAttr{ID}{Empleado}{ID}{Numero de registro para identificar al encargado de área}{SI}
	\brAttr{ID}{Sucursal}{ID}{Numero de registro para identificar la sucursal a la que pertenece}{SI}
	\brAttr{Capacidad}{Capacidad}{Numerico}{Numero de personas que pueden estar en el area}{SI}
	\brAttr{Dimensiones}{Dimensiones}{Nombre corto}{Metros de largo por metros de ancho}{SI}
	\cdtEntityRelSection
	\brRel{\brRelAgregation}{Encargado}{Un \hyperlink{Area}{Area} tiene un \hyperlink{Encargado}{Encargado} asignada}	
	\brRel{\brRelComposition}{Sucursal}{Un \hyperlink{Area}{Area} pertenece a una \hyperlink{Sucursal}{Sucursal}}	
	\end{cdtEntidad}
	
	%---------------------------------------------------------sa
	
	
	\section{Modelo de procesos}
	
	Los nuevos procesos se presentan en esta sección.
	
	\begin{figure}[htbp]
		\begin{center}
			\includegraphics[width=1.1\textwidth]{images/Procesos/agregarActividadSucursal}
			\caption{Agregar Actividad a Sucursal}
		\end{center}
	\end{figure}

\begin{figure}[htbp]
	\begin{center}
		\includegraphics[width=1.1\textwidth]{images/Procesos/IngresarSucursal}
		\caption{Ingreso a la sucursal}
	\end{center}
\end{figure}

\begin{figure}[htbp]
	\begin{center}
		\includegraphics[width=1.1\textwidth]{images/Procesos/registrarCliente}
		\caption{Registrar cliente}
	\end{center}
\end{figure}

\begin{figure}[htbp]
	\begin{center}
		\includegraphics[width=1.1\textwidth]{images/Procesos/ReponerTarjetaAcceso}
		\caption{Reposicion de Tarjeta}
	\end{center}
\end{figure}

\begin{figure}[htbp]
	\begin{center}
		\includegraphics[width=1.1\textwidth]{images/Procesos/VenderActividadCliente}
		\caption{Vender Actividad}
	\end{center}
\end{figure}

\begin{figure}[htbp]
	\begin{center}
		\includegraphics[width=1.1\textwidth]{images/Procesos/VenderActividadInvitado}
		\caption{Agregar Actividad a Invitado}
	\end{center}
\end{figure}

\begin{figure}[htbp]
	\begin{center}
		\includegraphics[width=1.1\textwidth]{images/Procesos/VenderMembresia}
		\caption{Vender Membresía}
	\end{center}
\end{figure}








