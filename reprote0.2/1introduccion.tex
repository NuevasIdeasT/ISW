%=========================================================
\chapter{Introducción}


%\cdtInstrucciones{
%	Presentar el documento, indicando su contenido, a quien va dirigido, quien lo realizó, por que razón, dónde y cuando. \\
%}
	Este documento contiene la Especificacion del proyecto ``{\em Work out}'' correspondiente al trabajo realizado en el semestre 2017-2018-2 para la materia de Ingeniería de software en el grupo 3CV9 por el equipo {\em Nuevas Ideas Tecnológicas (NIT)}.

%---------------------------------------------------------
\section{Presentación}


%\cdtInstrucciones{
%	Indique el propósito del documento y las distintas formas en que puede ser utilizado.\\
%}
	Este documento contiene la especificación de los requerimientos del usuario y del sistema a desarrollar. Tiene como objetivo establecer las funciones del sistema para la evaluación al final del semestre y así poder acreditar la materia. 
	
	Este documento es la documentación del proyecto presentado al final del semestre. ``{\em Work out}''.
	
%---------------------------------------------------------
\section{Organización del contenido}

	En el capítulo \ref{cap:reqUsr} se modela el alcance del sistema. Se presentan inicialmente los Actores involucrados y sus requerimientos. Después se presentan los requerimientos funcionales se presenta el modelo Físico y Lógico del sistema.
	
	En el capítulo \ref{cap:reqSist} se modela la {\em Arquitectura del negocio} la cual está conformada por los ({\em Términos} y {\em Hechos del negocio}), Arquitectura de procesos y las {\em Reglas del negocio}. Primero se especifica brevemente el {\em Contexto} en el que los términos tienen significado.
	
	En el capitulo \ref{cap:modDinamico} se detallan todos los escenarios de ejecución del sistema. La figura~\ref{fig:CUcompleto1}  y la figurara~\ref{fig:CUcompleto2} muestra todos los casos de uso del sistema. En este documento solo detallamos los casos de uso del proyecto de gymnasios.
	
	
	

%---------------------------------------------------------
\section{Notación, símbolos y convenciones utilizadas}

	Los requerimientos funcionales utilizan una clave RFX, donde:
	
\begin{description}
	\item[X] Es un número consecutivo: 1, 2, 3, ...
	\item[RF] Es la clave para todos los {\bf R}equerimientos {\bf F}uncionales.
\end{description}

	Los requerimientos del usuario utilizan una clave RUX, donde:
	
\begin{description}
	\item[X] Es un número consecutivo: 1, 2, 3, ...
	\item[RU] Es la clave para todos los {\bf R}equerimientos del {\bf U}suario.
\end{description}

	Además, para los requerimeitnos funcionales se usan las abreviaciones que se muestran en la tabla~\ref{tbl:leyendaRF}.
\begin{table}[hbtp!]
	\begin{center}
    \begin{tabular}{|r l|}
	    \hline
    	{\footnotesize Id} & {\footnotesize\em Identificador del requerimiento.}\\
    	{\footnotesize Pri.} & {\footnotesize\em Prioridad}\\
    	{\footnotesize Ref.} & {\footnotesize\em Referencia a los Requerimientos de usuario.}\\
    	{\footnotesize MA} & {\footnotesize\em Prioridad Muy Alta.}\\
    	{\footnotesize A} & {\footnotesize\em Prioridad Alta.}\\
    	{\footnotesize M} & {\footnotesize\em Prioridad Media.}\\
    	{\footnotesize B} & {\footnotesize\em Prioridad Baja.}\\
    	{\footnotesize MB} & {\footnotesize\em Prioridad Muy Baja.}\\
		\hline
    \end{tabular} 
    \caption{Leyenda para los requerimientos funcionales.}
    \label{tbl:leyendaRF}
	\end{center}
\end{table}