% \IUref{IUAdmPS}{Administrar Planta de Selección}
% \IUref{IUModPS}{Modificar Planta de Selección}
% \IUref{IUEliPS}{Eliminar Planta de Selección}

% 


% Copie este bloque por cada caso de uso:
%-------------------------------------- COMIENZA descripción del caso de uso.

%\begin{UseCase}[archivo de imágen]{UCX}{Nombre del Caso de uso}{
%--------------------------------------
	\begin{UseCase}{CU36}{Modificar Instructor}{
		Poder modificar los datos personales de un instructor y sus actividades asignadas.
	}
		\UCitem{Versión}{\color{Gray}0.4}
		\UCitem{Autor}{\color{Gray}Quiroz Olmedo Luis Eduardo}
		\UCitem{Supervisa}{\color{Gray}Francisco}
		\UCitem{Actor}{Gerente}
		\UCitem{Propósito}{Que los datos ingresados al momento de dar de alta un instructor puedan modificarse, así como también poder asignar nuevas actividades o editar las existentes asignadas a un instructor.}
		\UCitem{Entradas}{Instructor}
		\UCitem{Origen}{Teclado}
		\UCitem{Salidas}{Mensaje de confirmación}
		\UCitem{Destino}{Pantalla de Instructores.}
		\UCitem{Precondiciones}{ •Que el actor tenga una sesión iniciada en el sistema.\newline
        • Que el sistema tenga al menos un instructor y una actividad registrada.
}
		\UCitem{Postcondiciones}{El Instructor seleccionado ha sido modificado y se ve el cambio en la pantalla de Intructores}
		\UCitem{Errores}{
		 • El actor dejo los campos Nombre y Dirección completamente vacíos.\textbf{Trayectoria Alternativa B}\newline
		 El actor dejo el campo de “Actividades Asignadas” sin ninguna actividad.\textbf{Trayectoria Alternativa C}.
		}
		\UCitem{Tipo}{Caso de uso primario}
		\UCitem{Observaciones}{}
	\end{UseCase}
%--------------------------------------
	\begin{UCtrayectoria}{Este caso de uso inicia cuando un actor se encuentra en la pantalla de Instructores "Poner aqui la IU" y tiene una sesión iniciada.}
		\UCpaso[\UCactor] selecciona el icono de edición sobre el instructor que se va a modificar.
		\UCpaso despliega la pantalla “Editar instructor”.
		\UCpaso permite editar los campos de texto “Nombre” y “Dirección”.
		\UCpaso[\UCactor] introduce un Nombre y Dirección nuevos.
		\UCpaso consulta las actividades con horarios libres y las despliega en la pantalla en forma de lista.
		\UCpaso[\UCactor] selecciona una actividad.
		\UCpaso consulta los horarios disponibles de la actividad seleccionada y los despliega en la pantalla en forma de lista.
		\UCpaso[\UCactor] selecciona un horario.
		\UCpaso[\UCactor] selecciona el boton guardar.
		\UCpaso actualiza los campos de nombre y dirección con los valores introducidos en el paso 4.
		\UCpaso \textbf{ INCLUDE(CU38 Asignar Actividad de Instructor)}.
		\UCpaso despliega la pantalla de Instructores "IU" con el mensaje {\bf MSG4-}``El Instructor [{\em Nombre de Instructor}] ha sido actualizado con éxito.''.
	\end{UCtrayectoria}

%--------------------------------------		
		\begin{UCtrayectoriaA}{A}{El actor NO introduce un nombre o dirección nuevos.}
			\UCpaso deja los campos “Nombre” y “Dirección” sin ningún cambio.
		\end{UCtrayectoriaA}
		
%--------------------------------------
		\begin{UCtrayectoriaA}{B}{El actor deja vacío el campo “Nombre” o “Dirección” o ambos campos.}
			\UCpaso despliega un mensaje dentro del campo con la leyenda {\bf MSG5-}``Campo Requerido'' .
		\end{UCtrayectoriaA}
%--------------------------------------
		\begin{UCtrayectoriaA}{C}{El actor NO selecciona ninguna actividad.}
			\UCpaso despliega un mensaje en pantalla con la leyenda {\bf MSG6-}``Asignar al menos una actividad'' .
		\end{UCtrayectoriaA}
%--------------------------------------
		\begin{UCtrayectoriaA}{D}{El actor selecciona el icono de borrado de la actividad }
			\UCpaso muestra una alerta en pantalla con el mensaje “Seguro que deseas eliminar la actividad” y las opciones “Aceptar” y “Cancelar”.
		    \UCpaso[\UCactor] selecciona la opción Aceptar.
		    \UCpaso elimina la relacion de con el horario y la actividad y la elimina de la lista de actividades asignadas.
		\end{UCtrayectoriaA}
%--------------------------------------
		\begin{UCtrayectoriaA}{E}{El actor selecciona la pestaña “Nueva Actividad”.}
			\UCpaso ejecuta los pasos [5-12].
		\end{UCtrayectoriaA}
%--------------------------------------
		\begin{UCtrayectoriaA}{F}{El sistema no tiene actividades disponibles}
			\UCpaso Muestra una alerta en pantalla con el mensaje "No hay actividades disponibles".
		\end{UCtrayectoriaA}		

% Copie este bloque por cada caso de uso:
%-------------------------------------- COMIENZA descripción del caso de uso.

%\begin{UseCase}[archivo de imágen]{UCX}{Nombre del Caso de uso}{
%--------------------------------------
\begin{UseCase}{CU38}{Registrar Instructor}{
		Crear un nuevo instructor en el sistema introduciendo sus datos personales y asignándole al menos una actividad.
	}
	\UCitem{Versión}{\color{Gray}0.4}
	\UCitem{Autor}{\color{Gray}Quiroz Olmedo Luis Eduardo}
	\UCitem{Supervisa}{\color{Gray}Francisco}
	\UCitem{Actor}{Gerente}
	\UCitem{Propósito}{Permitir al Gerente registrar nuevos Instructores en el sistema para que se les pueda asignar actividades nuevas o existentes y poder gestionar a los instructores.}
	\UCitem{Entradas}{•Nombre\newline
	•Dirección\newline
	•Nombre de Usuario\newline
	•Password}
	\UCitem{Origen}{Mouse y Teclado}
	\UCitem{Salidas}{Mensaje de confirmación.}
	\UCitem{Destino}{Pantalla de Instructores.}
	\UCitem{Precondiciones}{El jefe de inmobiliario debe inciiar sesion para ver las actividades, debe haber al menos una actividad.}
	\UCitem{Postcondiciones}{•Que el actor tenga una sesión iniciada
	\newline•Que el nombre de Usuario no este dado de alta en el sistema}
	\UCitem{Errores}{•Campos de Texto nulos o incompletos.\textbf{Trayectoria Alternativa A y B} \newline
	•Todas las actividades ocupadas.\textbf{Trayectoria Alternativa C}\newline
	•El actor no selecciona ninguna actividad.\textbf{Trayectoria Alternativa D}\newline}
	\UCitem{Tipo}{Caso de uso primario}
	\UCitem{Observaciones}{Todos los campos del formulario son obligatorios.}
\end{UseCase}
%--------------------------------------
\begin{UCtrayectoria}{Este caso de uso inicia cuando un actor se encuentra en la pantalla de “Instructores”}
	\UCpaso[\UCactor] selecciona el boton "Agregar Nuevo Instructor".
	\UCpaso despliega la pantalla "UI" "Nuevo Instructor"
	\UCpaso[\UCactor]introduce el nombre completo del instructor en el campo de texto “Nombre completo”.
	\UCpaso sistema valida que los valores introducidos cumplen con las caracteristicas de un nombre.[aA-zA]
	\UCpaso[\UCactor]introduce la dirección del instructor en el campo de texto “Dirección”.
	\UCpaso sistema valida que los valores introducidos cumplen con las caracteristicas de una dirección.[aA-zA1-9 '.' '-']
	\UCpaso selecciona el boton "Nueva Actividad"
	\UCpaso consulta que actividades tienen horarios libres y las despliega en la pantalla en forma de lista.
	\UCpaso[\UCactor] selecciona una actividad.
	\UCpaso \textbf{INCLUDE (CU38 Asignar Actividad al Instructor)}.
	\UCpaso[\UCactor] da clic en el icono guardar.
	\UCpaso valida que al menos una actividad ha sido asignada.
	\UCpaso crea una nueva entidad Instructor con los datos introducidos ligado con las o la actividad seleccionadas y sus horarios seleccionados.
	\UCpaso despliega la pantalla de instructores con el mensaje {\bf MSG7-} “Registro Exitoso”.
\end{UCtrayectoria}
%--------------------------------------
\begin{UCtrayectoriaA}{A}{El campo ''Nombre completo'' tiene menos de dos palabras o es nulo.}
			\UCpaso despliega un mensaje dentro del campo con la leyenda “Nombre invalido” y limpia el campo “Nombre Completo” y se continúa en el paso [1].
\end{UCtrayectoriaA}
%--------------------------------------
\begin{UCtrayectoriaA}{B}{El campo ''Direción'' tiene menos de dos palabras o es nulo.}
			\UCpaso despliega un mensaje dentro del campo con la leyenda “Dirección invalida” y limpia el campo “Direccion” y se continúa en el paso [1].
\end{UCtrayectoriaA}
%--------------------------------------
\begin{UCtrayectoriaA}{C}{Todas las Actividades se encuentran ocupadas}
		   \UCpaso despliega un mensaje en pantalla con la leyenda {\bf MSG8-} “No se cuenta con Actividades disponibles para asignar”.
		   \UCpaso despliega la pantalla de Instructores cancelando el registro y no guardando ningún dato.
\end{UCtrayectoriaA}
%--------------------------------------
\begin{UCtrayectoriaA}{D}{El actor no selecciono ninguna actividad}
		   \UCpaso despliega un mensaje en pantalla con la leyenda {\bf MSG9-} “Se debe seleccionar al menos una actividad” y se continua en el paso [9].	  
\end{UCtrayectoriaA}
%--------------------------------------
\begin{UCtrayectoriaA}{E}{El actor selecciona el boton ''Cancelar''}
		   \UCpaso muestra una alerta en pantalla con la leyenda {\bf MSG10-} “Estas seguro que deseas Cancelar” y dando las opciones Aceptar y Cancelar.
		   \UCpaso[\UCactor] selecciona ''Aceptar''.
		   \UCpaso despliega la pantalla de ''UI'' Instructores cancelando el registro y no guardando ningún dato.
\end{UCtrayectoriaA}

%--------------------------------------

%\begin{UseCase}[archivo de imágen]{UCX}{Nombre del Caso de uso}{
%--------------------------------------
\begin{UseCase}{CU38}{Asignar Actividad al Instructor}{
	Asignar una entidad Actividad seleccionada previamente con una entidad Instructor.
	}
	\UCitem{Versión}{\color{Gray}0.4}
	\UCitem{Autor}{\color{Gray}Quiroz Olmedo Luis}
	\UCitem{Supervisa}{\color{Gray}Francisco}
	\UCitem{Actor}{Gerente}
	\UCitem{Propósito}{Crear una unión con una actividad seleccionada y un horario y para evitar duplicados en asignaciones de actividades}
	\UCitem{Entradas}{Actividad}
	\UCitem{Origen}{Mouse y teclado}
	\UCitem{Salidas}{-}
	\UCitem{Destino}{Pantalla de Editar Instructor o Registrar Instructor}
	\UCitem{Precondiciones}{• Que el sistema tenga al menos un instructor y una actividad registrados.\newline
	• Que el actor haya seleccionado una actividad.
	}
	\UCitem{Postcondiciones}{Quedara unido un horario de la Actividad seleccionada con un Instructor.}
	\UCitem{Errores}{Formulario no este llenado de acuerdo al formato.}
	\UCitem{Tipo}{Caso de uso secundario}
	
\end{UseCase}
%--------------------------------------
\begin{UCtrayectoria}{Este caso de uso inicia cuando un actor se encuentra en la pantalla de ''Nuevo Instructor'' o en la pantalla ''Editar Instructor'' y seleccionó una actividad disponible}
	\UCpaso consulta los horarios que no han sido asignados de la actividad seleccionada y los despliega en la pantalla en forma de lista.
	\UCpaso[\UCactor] selecciona el horario.
	\UCpaso el sistema elimina de la lista de horarios disponibles el horario seleccionado.
	\UCpaso el sistema reserva el horario para que ya no sea mostrado en alguna sesion existente.
	\UCpaso el sistema liga el horario seleccionado con el Instructor que se esta editando.
\end{UCtrayectoria}

%\BRref{BR129}{Determinar si un Estudiante puede inscribir Seminario.} \Trayref{A}
%--------------------------------------		
\begin{UCtrayectoriaA}{A}{No hay horarios disponibles de la actividad seleccionada.}
	\UCpaso muestra un mensaje en pantalla con la leyenda {\bf MSG11-} ''Sin horarios disponibles''.
	\UCpaso redirecciona ala pantalla \IUref{IU3}{Editar Instructor} o  \IUref{IU4}{Registrar Instructor} dependiendo de donde venga la selección.
	%\UCpaso[\UCactor] Continua con el paso 8 del \UCref{CU3}.
\end{UCtrayectoriaA}
%--------------------------------------		
\begin{UCtrayectoriaA}{B}{El actor no selecciona ningún horario}

	\UCpaso muestra un mensaje en pantalla con la leyenda {\bf MSG12-} ''Se debe seleccionar al menos un horario'' y se continua en el paso [2].
\end{UCtrayectoriaA}


% Copie este bloque por cada caso de uso:
%-------------------------------------- COMIENZA descripción del caso de uso.

%\begin{UseCase}[archivo de imágen]{UCX}{Nombre del Caso de uso}{
%--------------------------------------
\begin{UseCase}{CU4}{Ver actividad de sucursal}{
		El sistema muestrará la información completa de una actividad seleccionada.
	}
	\UCitem{Versión}{\color{Gray}0.4}
	\UCitem{Autor}{\color{Gray}Jazmin Camarillo Martínez}
	\UCitem{Supervisa}{\color{Gray}Francisco}
	\UCitem{Actor}{Jefe de inmobiliario}
	\UCitem{Propósito}{Para que el jefe de inmobiliario pueda revisar que los datos de la actividad sean correctos.}
	\UCitem{Entradas}{botón de ver actividad.}
	\UCitem{Origen}{Teclado}
	\UCitem{Salidas}{Numero de actividad, nombre de la actividad, nombre de la sucursal, el dia y la hora que se imparte la actividad,nombre del instructor, la descripcion de la actividad, dirección de la sucursal, imagen(es).}
	\UCitem{Destino}{Pantalla del jefe de inmobiliario.}
	\UCitem{Precondiciones}{El jefe de inmobiliario debe estar dentro de susesión para poder ver la información. La actividad que desee ver, debe estar registrada.}
	\UCitem{Postcondiciones}{}
	\UCitem{Errores}{Actividad inexistente}
	\UCitem{Tipo}{Caso de uso primario}
	\UCitem{Observaciones}{}
\end{UseCase}
%--------------------------------------
\begin{UCtrayectoria}{Principal}
	\UCpaso[\UCactor] Ingresa a la \IUref{IU3}{Pantalla de Actividades.}\label{CU1LoginJI}.
	\UCpaso[\UCactor] Oprime el botón de ver.
	\UCpaso Despliega la \IUref{IU6}{Pantalla Ver Actividad}.
	\UCpaso Oprime el botón de aceptar para regresar a la  \IUref{IU3}{Pantalla de Actividades.}\label{CU1LoginJI}.
\end{UCtrayectoria}


%--------------------------------------
% Puntos de extensión
\subsection{Puntos de extensión}
\UCExtenssionPoint{
	% Cuando:
	Desea modificar datos de actividad.
}{
	% Durante la región:
	En el paso 3.
}{
	% Casos de uso a los que extiende:
	\hyperlink{CU3}{CU3 Modificar actividad de sucursal}.
}



% Copie este bloque por cada caso de uso:
%-------------------------------------- COMIENZA descripción del caso de uso.

%\begin{UseCase}[archivo de imágen]{UCX}{Nombre del Caso de uso}{
%--------------------------------------
\begin{UseCase}{CU4}{Agregar actividad a la sucursal}{
		El Jefe de inmobiliario podrá agregar una nueva actividad al sistema.
	}
	\UCitem{Versión}{\color{Gray}0.4}
	\UCitem{Autor}{\color{Gray}Jazmin Camarillo Martínez}
	\UCitem{Supervisa}{\color{Gray}Francisco}
	\UCitem{Actor}{Jefe de inmobiliario}
	\UCitem{Propósito}{Para que cada que se agregue una nueva actividad a la sucursal se pueda incorporar su información al sistema.}
	\UCitem{Entradas}{}
	\UCitem{Origen}{Teclado}
	\UCitem{Salidas}{.}
	\UCitem{Destino}{Pantalla del jefe de inmobiliario.}
	\UCitem{Precondiciones}{.}
	\UCitem{Postcondiciones}{El jefe de inmobiliario deberá ver un menú con; perfil, sucursales, actividades y áreas, y sus datos de salida. }
	\UCitem{Errores}{}
	\UCitem{Tipo}{Caso de uso primario}
	\UCitem{Observaciones}{}
\end{UseCase}
%--------------------------------------
\begin{UCtrayectoria}{Principal}
	\UCpaso[\UCactor] Introduce su Nombre de usuario y su password para poder ingresar vía la  \IUref{IU1}{Pantalla de Inicio de Sesión del Jefe de Inmobiliario.}\label{CU1LoginJI}.
	\UCpaso[\UCactor] Confirma la operación presionando el botón Iniciar Sesión.
	\UCpaso Verifica que el nombre y password son correctos para ingresar sesión del jefe de inmobiliario.
	\UCpaso Despliega la \IUref{IU2}{Pantalla inicio del jefe de inmobiliario}.
\end{UCtrayectoria}

%--------------------------------------		
\begin{UCtrayectoriaA}{A}{El nombre de sesión no existe}
	\UCpaso[\UCactor] Muestra el Mensaje {\bf MSG1-}``El empleado [{\em Nombre de sesión}] no existe.''.
	\UCpaso[\UCactor] Introduce nombre usuario correcto.
	\UCpaso[] Continua con el paso 3 del \UCref{CU1}.
\end{UCtrayectoriaA}

%--------------------------------------
\begin{UCtrayectoriaA}{B}{Contraseña incorrecta}
	\UCpaso Muestra el Mensaje {\bf MSG1¿2-}``El password no es correcto.''.
	\UCpaso[\UCactor] Introduce password correcto.
	\UCpaso[] Continua con el paso 3 del \UCref{CU1}.
\end{UCtrayectoriaA}


%--------------------------------------
% Puntos de extensión
\subsection{Puntos de extensión}
\UCExtenssionPoint{
	% Cuando:
	Desea acceder a las sucursales.
	Desea acceder a las actividades.
	Desea acceder a las áreas.
	Desea acceder a su perfil.
}{
	% Durante la región:
	En el paso 5.
}{
	% Casos de uso a los que extiende:
	\hyperlink{CU2}{CU2 Consultar Actividades de sucursal}.
}
