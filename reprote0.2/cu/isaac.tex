% \IUref{IUAdmPS}{Administrar Planta de Selección}
% \IUref{IUModPS}{Modificar Planta de Selección}
% \IUref{IUEliPS}{Eliminar Planta de Selección}

% Copie este bloque por cada caso de uso:
%-------------------------------------- COMIENZA descripción del caso de uso.

%\begin{UseCase}[archivo de imágen]{UCX}{Nombre del Caso de uso}{
%--------------------------------------
	\begin{UseCase}{CU17}{Consultar Datos de Instructor.}{
			El recepcionista podrá acceder al sistema para corroborar el horario y clases que imparte 
		un instructor en el gimnasio.}
		\UCitem{Versión}{\color{Gray}0.4}
		\UCitem{Autor}{\color{Gray}Andrés Isaac García Martínez}
		\UCitem{Supervisa}{\color{Gray}Jazmín Camarillo Martínez}
		\UCitem{Actor}{Recepcionista}
		\UCitem{Propósito}{Que tanto el cliente como el instructor puedan ver los horarios y clases de los
		instructores contratados en el gimnasio.}
		\UCitem{Entradas}{Nombre, Apellido Paterno, Apellido Materno.}
		\UCitem{Origen}{Teclado.}
		\UCitem{Salidas}{Nombre completo del instructor, Edad, Clases que imparte, Horario, Imágen del instructor.}
		\UCitem{Destino}{Pantalla del recepcionista.}
		\UCitem{Precondiciones}{El recepcionista tiene que iniciar sesión.}
		\UCitem{Postcondiciones}{-. }
		\UCitem{Errores}{}
		\UCitem{Tipo}{Caso de uso primario}
		\UCitem{Observaciones}{}
	\end{UseCase}
%--------------------------------------
	\begin{UCtrayectoria}{Principal}
		\UCpaso[\UCactor] Introduce su Nombre de usuario y su password para poder ingresar vía la  \IUref{IU1}{Pantalla de Inicio de Sesión del Recepcionista.}\label{CU1LoginJI}.
		\UCpaso[\UCactor] Confirma la operación presionando el botón Iniciar Sesión.
		\UCpaso Verifica que el nombre y contraseña para ingresar iniciar sesión se encuentren.
		\UCpaso Despliega la \IUref{IU2}{Pantalla inicio del jefe de inmobiliario}.
		\UCpaso Seleccciona Consultar Instructor.
		\UCpaso Despliega la \IUref{IU17}{Pantalla de Consultar Instructor}
		\UCpaso[\UCactor]Ingresa en el área de texto designada el nombre del instructor, así como los apellidos paterno y materno
				en sus áreas de texto designadas.
		\UCpaso Búsca coincidencias con los datos ingresados.
		\UCpaso Despliega una pantalla con el nombre completo del instructor, Edad, Clases que imparte, horario y una foto del instructor.

	\end{UCtrayectoria}

%--------------------------------------		
		\begin{UCtrayectoriaA}{A}{El usuario no existe}
			\UCpaso[\UCactor] Muestra el Mensaje {\bf MSG1-}``El empleado [{\em Nombre de sesión}] no existe.''.
			\UCpaso[\UCactor] Introduce nombre usuario correcto.
			\UCpaso[] Continua con el paso 3 del \UCref{CU1}.
		\end{UCtrayectoriaA}
		
%--------------------------------------
		\begin{UCtrayectoriaA}{B}{Contraseña incorrecta}
			\UCpaso Muestra el Mensaje {\bf MSG1¿2-}``La contraseña no es correcta.''.
			\UCpaso[\UCactor] Introduce la contraseña correcta.
			\UCpaso[] Continua con el paso 3 del \UCref{CU1}.
		\end{UCtrayectoriaA}


%--------------------------------------		
		
% Copie este bloque por cada caso de uso:
%-------------------------------------- COMIENZA descripción del caso de uso.

%\begin{UseCase}[archivo de imágen]{UCX}{Nombre del Caso de uso}{
%--------------------------------------
\begin{UseCase}{CU18}{Consultar Datos de Sucursal.}{
			Que el recepcionista acceda para poder corroborar, e informar al cliente sobre horarios de atención, las distintas áreas con
las que cuenta la sucursal así como datos generales de la sucursal}
		\UCitem{Versión}{\color{Gray}0.4}
		\UCitem{Autor}{\color{Gray}Andrés Isaac García Martínez}
		\UCitem{Supervisa}{\color{Gray}Jazmín Camarillo Martínez}
		\UCitem{Actor}{Recepcionista}
		\UCitem{Propósito}{Que el recepcionista proporcione información correcta sobre la sucursal en cuestión.}
		\UCitem{Entradas}{Ubicación de la sucursal.}
		\UCitem{Origen}{Teclado.}
		\UCitem{Salidas}{Ubicación exacta de la sucursal, horarios de atención, Áreas con las que cuenta, Actividades que se imparten}
		\UCitem{Destino}{Pantalla del recepcionista.}
		\UCitem{Precondiciones}{El recepcionista tiene que iniciar sesión.}
		\UCitem{Postcondiciones}{-. }
		\UCitem{Errores}{}
		\UCitem{Tipo}{Caso de uso primario}
		\UCitem{Observaciones}{}
	\end{UseCase}
%--------------------------------------
	\begin{UCtrayectoria}{Principal}
		\UCpaso[\UCactor] Introduce su Nombre de usuario y su password para poder ingresar vía la  \IUref{IU1}{Pantalla de Inicio de Sesión del Recepcionista.}\label{CU1LoginJI}.
		\UCpaso[\UCactor] Confirma la operación presionando el botón Iniciar Sesión.
		\UCpaso Revisa que el nombre y contraseña son correctos para ingresar como usuario recepcionista.
		\UCpaso Despliega la \IUref{IU2}{Pantalla inicio del jefe de inmobiliario}.
		\UCpaso[\UCactor]Selecciona sucursales.
		\UCpaso Despliega la \IUref{IU18}{Pantalla de Consultar Sucursal}
		\UCpaso[\UCactor]Ingresa la dirección de la sucursal.
		\UCpaso Busca coincidencias con los datos introducidos.
		\UCpaso Despliega la pantalla con la ubicacion de la sucursal, horarios de atención, áreas y actividades.

	\end{UCtrayectoria}

%--------------------------------------		
		\begin{UCtrayectoriaA}{A}{Ubicación inexacta}
			\UCpaso[\UCactor] Muestra el Mensaje {\bf MSG1-}``La ubicación [{\em Ubicación}] no existe.''.
			\UCpaso[\UCactor] Introduce la ubicación correcta.
			\UCpaso[] Continua con el paso 8 del \UCref{CU18}.
		\end{UCtrayectoriaA}
		
%--------------------------------------	

% Copie este bloque por cada caso de uso:
%-------------------------------------- COMIENZA descripción del caso de uso.

%\begin{UseCase}[archivo de imágen]{UCX}{Nombre del Caso de uso}{
%--------------------------------------
\begin{UseCase}{CU19}{Consultar Datos de Actividades.}{Acceder al sistema como usuario Recepcionista para 
poder ver la información principal de cada una de las actividades que se imparten en la sucursal}
		\UCitem{Versión}{\color{Gray}0.4}
		\UCitem{Autor}{\color{Gray}Andrés Isaac García Martínez}
		\UCitem{Supervisa}{\color{Gray}Jazmín Camarillo Martínez}
		\UCitem{Actor}{Recepcionista}
		\UCitem{Propósito}{Que el recepcionista pueda proporcionae la información referente a cada una 
		de las actividades o clases que se dan en el gimnasio. .}
		\UCitem{Entradas}{Nombre de la actividad.}
		\UCitem{Origen}{Teclado.}
		\UCitem{Salidas}{Nombre de la actividad, horarios en los que se imparte, área donde se imparte, Instructor que la imparte.}
		\UCitem{Destino}{Pantalla del recepcionista.}
		\UCitem{Precondiciones}{El recepcionista tiene que iniciar sesión.}
		\UCitem{Postcondiciones}{-. }
		\UCitem{Errores}{}
		\UCitem{Tipo}{Caso de uso primario}
		\UCitem{Observaciones}{}
	\end{UseCase}
%--------------------------------------
	\begin{UCtrayectoria}{Principal}
		\UCpaso[\UCactor] Introduce su Nombre de usuario y su password para poder ingresar vía la  \IUref{IU1}{Pantalla de Inicio de Sesión del Recepcionista.}\label{CU1LoginJI}.
		\UCpaso[\UCactor] Confirma la operación presionando el botón Iniciar Sesión.
		\UCpaso Revisa que el nombre y contraseña son correctos para ingresar sesión como recepcionista.
		\UCpaso Despliega la \IUref{IU2}{Pantalla inicio del jefe de inmobiliario}.
		\UCpaso[\UCactor]Selecciona actividades.
		\UCpaso Despliega la \IUref{IU19}{Pantalla de Consultar Actividades}
		\UCpaso[\UCactor]Selecciona la actividad que desea consultar así como el día en el que se imparte.
		\UCpaso Despliega la pantalla con el nombre de la actividad, horarios, días, área donde se imparte así como el instructor.

	\end{UCtrayectoria}

%--------------------------------------		
		\begin{UCtrayectoriaA}{A}{Actividad mal escrito.}
			\UCpaso[\UCactor] Muestra el Mensaje {\bf MSG10-}``La actividad [{\em Nombre de Actividad}] no existe.''.
			\UCpaso[\UCactor] Selecciona la actividad correcta..
			\UCpaso[] Continua con el paso 5 del \UCref{CU19}.
		\end{UCtrayectoriaA}
			
%--------------------------------------	

% Copie este bloque por cada caso de uso:
%-------------------------------------- COMIENZA descripción del caso de uso.

%\begin{UseCase}[archivo de imágen]{UCX}{Nombre del Caso de uso}{
%--------------------------------------
\begin{UseCase}{CU20}{Consultar Datos de Áreas.}{Apoyo para la recepción para proporcionar información a todo cliente que visite la sucursal.}
		\UCitem{Versión}{\color{Gray}0.4}
		\UCitem{Autor}{\color{Gray}Andrés Isaac García Martínez}
		\UCitem{Supervisa}{\color{Gray}Jazmín Camarillo Martínez}
		\UCitem{Actor}{Recepcionista}
		\UCitem{Propósito}{Que el recepcionista pueda informar con mayor certeza al cliente sobre las áreas de la sucursal. }
		\UCitem{Entradas}{Nombre del área.}
		\UCitem{Origen}{Teclado.}
		\UCitem{Salidas}{Nombre del área, capacidad mínima y máxima, Dimensiones.}
		\UCitem{Destino}{Pantalla del recepcionista.}
		\UCitem{Precondiciones}{El recepcionista tiene que iniciar sesión.}
		\UCitem{Postcondiciones}{-. }
		\UCitem{Errores}{}
		\UCitem{Tipo}{Caso de uso primario}
		\UCitem{Observaciones}{}
	\end{UseCase}
%--------------------------------------
	\begin{UCtrayectoria}{Principal}
		\UCpaso[\UCactor] Introduce su Nombre de usuario y su password para poder ingresar vía la  \IUref{IU1}{Pantalla de Inicio de Sesión del Recepcionista.}\label{CU1LoginJI}.
		\UCpaso[\UCactor] Confirma la operación presionando el botón Iniciar Sesión.
		\UCpaso Revisa que el nombre y contraseña son correctos para ingresar sesión como recepcionista.
		\UCpaso Despliega la \IUref{IU2}{Pantalla inicio de recepcionosta}.
		\UCpaso[\UCactor]Selecciona actividades.
		\UCpaso Despliega la \IUref{IU20}{Pantalla de Consultar Áreas}
		\UCpaso[\UCactor]Selecciona el área que desea revisar.
		\UCpaso Busca coincidencias con el nombre del área que se ingreso.
		\UCpaso Despliega la pantalla con el Nombre del área, sus capacidades máxima y mínima.
	\end{UCtrayectoria}
		
%--------------------------------------	

% Copie este bloque por cada caso de uso:
%-------------------------------------- COMIENZA descripción del caso de uso.

%\begin{UseCase}[archivo de imágen]{UCX}{Nombre del Caso de uso}{
%--------------------------------------
\begin{UseCase}{CU21}{Consultar Datos de Clientes.}{El recepcionista podrá acceder al sistema para poder corroborar los datos de los clientes.}
		\UCitem{Versión}{\color{Gray}0.4}
		\UCitem{Autor}{\color{Gray}Andrés Isaac García Martínez}
		\UCitem{Supervisa}{\color{Gray}Jazmín Camarillo Martínez}
		\UCitem{Actor}{Recepcionista}
		\UCitem{Propósito}{Que el recepcionista pueda corrobar la información del cliente. }
		\UCitem{Entradas}{Correo eletrcónico, Nombre(s), Apellido Paterno, Apellido Materno.}
		\UCitem{Origen}{Teclado.}
		\UCitem{Salidas}{Nombre completo del cliente, Edad, Peso, actividades en las que participa, Tipo de membresía, Fecha de inscripción.}
		\UCitem{Destino}{Pantalla del recepcionista.}
		\UCitem{Precondiciones}{El recepcionista tiene que iniciar sesión.}
		\UCitem{Postcondiciones}{-. }
		\UCitem{Errores}{}
		\UCitem{Tipo}{Caso de uso primario}
		\UCitem{Observaciones}{}
	\end{UseCase}
%--------------------------------------
	\begin{UCtrayectoria}{Principal}
		\UCpaso[\UCactor] Introduce su Nombre de usuario y su password para poder ingresar vía la  \IUref{IU1}{Pantalla de Inicio de Sesión del Recepcionista.}\label{CU1LoginJI}.
		\UCpaso[\UCactor] Confirma la operación presionando el botón Iniciar Sesión.
		\UCpaso Revisa que el nombre y contraseña son correctos para ingresar sesión como recepcionista.
		\UCpaso Despliega la \IUref{IU2}{Pantalla inicio de recepcionosta}.
		\UCpaso[\UCactor]Selecciona actividades.
		\UCpaso Despliega la \IUref{IU21}{Pantalla para consultar datos de clientes.}
		\UCpaso[\UCactor]Ingresa el correo electrónico del cliente.
		\UCpaso Busca coincidencias con el correo que se ingreso.
		\UCpaso Despliega la pantalla con el nombre completo del clinete, Edad, Peso, actividades en las que participa, Tipo de membresía y la Fecha de inscripción.
	\end{UCtrayectoria}

%--------------------------------------		
		\begin{UCtrayectoriaA}{A}{El cliente no se acuerda de su correo electrónico.}
			\UCpaso[\UCactor] Ingresa el nombre y apellidos del cliente.
			\UCpaso Continua desde el paso 9.
		\end{UCtrayectoriaA}