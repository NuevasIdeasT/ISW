% \IUref{IUAdmPS}{Administrar Planta de Selección}
% \IUref{IUModPS}{Modificar Planta de Selección}
% \IUref{IUEliPS}{Eliminar Planta de Selección}

% 


% Copie este bloque por cada caso de uso:

%-------------------------------------- COMIENZA descripción del caso de uso.

%\begin{UseCase}[archivo de imágen]{CUX}{Nombre del Caso de uso}{
%--------------------------------------
	\begin{UseCase}{CU17}{Consultar instructores de actividades.}{
		El cliente desde su inicio de sesión puede ver ir a una pantalla con una lista de  instructores que tiene asignado de acuerdo a las actividades en las que esta registrado.
	}
		\UCitem{Versión}{\color{Gray}0.2}
		\UCitem{Autor}{\color{Gray}Daniel.}
		\UCitem{Supervisa}{\color{Gray}Luis.}
		\UCitem{Actor}{\hyperlink{Cliente}{Cliente}}
		\UCitem{Propósito}{Que el cliente pueda conocer quienes son los instructores de las actividades a las que asiste y posteriormente poder calificarlos.}
		\UCitem{Entradas}{nombre de Instructor}
		\UCitem{Origen}{Mouse del cliente.}
		\UCitem{Salidas}{Nombre instructor,actividad, sucursal, calificación.}
		\UCitem{Destino}{Pantalla del cliente.}
		\UCitem{Precondiciones}{Que el cliente haya iniciado sesión, que el cliente tenga alguna actividad inscrita.}
		\UCitem{Postcondiciones}{}
		\UCitem{Errores}{Que el cliente no tenga ninguna actividad inscrita.}
		\UCitem{Tipo}{Caso de uso primario}
		\UCitem{Observaciones}{Los instructores serán mostrados de la siguiente manera:Nombre,edad, Clase, Sucursal,Calificación.}
	\end{UseCase}
%--------------------------------------
 observaciones
	\begin{UCtrayectoria}{Principal}
		\UCpaso[\UCactor] desde la pantalla de inicio presionara el botón de "mis Instructores".
		\UCpaso muestra la pantalla de los instructores.
		\UCpaso mostrara a los instructores en una tabla de acuerdo a como se especifica en las observaciones.
	\end{UCtrayectoria}


		
		
%-------------------------------------- TERMINA descripción del caso de uso.

%-------------------------------------- COMIENZA descripción del caso de uso.

%\begin{UseCase}[archivo de imágen]{CUX}{Nombre del Caso de uso}{
%--------------------------------------
	\begin{UseCase}{CU22}{Iniciar sesión cliente gimnasio}{
		Permite  al cliente del gimnasio autenticarse en el sistema mediante correo y su contraseña para poder ingresar a su cuenta.
	}
		\UCitem{Versión}{\color{Gray}0.2}
		\UCitem{Autor}{\color{Gray}Jazmin.}
		\UCitem{Supervisa}{\color{Gray}Paco.}
		\UCitem{Actor}{\hyperlink{Cliente}{Cliente}}
		\UCitem{Propósito}{Que el cliente pueda pueda acceder a su información, actividades que tiene inscritas o para poder inscribir alguna y ver sus instructores.}
		\UCitem{Entradas}{Correo electrónico, contraseña}
		\UCitem{Origen}{El actor mediante el teclado ingresara los datos mencionados}
		\UCitem{Salidas}{Nombre completo, Dirección, Calle, No. int, No. Ext. Col. Del./Mun. C.P, Peso, Enfermedades, Alergias, Tipo de sangre, Medicamentos.}
		\UCitem{Destino}{Pantalla del cliente.}
		\UCitem{Precondiciones}{Que el cliente este registrado en el sistema, que no haya una sesión ya activa.}
		\UCitem{Postcondiciones}{El cliente va ver su pantalla de inicio.}
		\UCitem{Errores}{Que el correo sea incorrecto, que la contraseña sea errónea, que no este registrado.}
		\UCitem{Tipo}{Caso de uso primario}
		\UCitem{Observaciones}{Este caso de uso tendrá pestañas adicionales para ver sus actividades, horarios e instructores que tiene a demás de poder inscribir actividad. A grandes rasgos este C.U. solo muestra los datos personales del cliente del gimnasio.}
	\end{UseCase}
%--------------------------------------

	\begin{UCtrayectoria}{Principal}
		\UCpaso[\UCactor] Ingresa su correo.  %\IUref{IU23}{Pantalla de Control de Acceso}\label{CU17Login}.
		\UCpaso[\UCactor] Ingresa su contraseña.%\BRref{BR129}{Determinar si un Estudiante puede inscribir Seminario.} \Trayref{A}.
		\UCpaso[\UCactor] Da clic en el botón de iniciar sesión\Trayref{A}\label{CU15ValidarQR}.  %\IUref{IU32}{Pantalla de Selección de Seminario}
		\UCpaso valida que el usuario y contraseña son correctas para iniciar sesión. 
		\UCpaso mostrara el nombre completo, su direccion y los datos medicos..% \BRref{BR130}{Determinar si un Estudiante puede inscribirse en un Seminario} \Trayref{C}.
		\UCpaso  mostrara en el menu superior el nombre del usuario y una lista deplegable ahi mismo con la opción de ver cursos, ver instructores y cerrrar sesión.\Trayref{C}\label{CU15RealizarPago}. %\BRref{BR143}{Validar el horario del estudiante} \Trayref{D}.		
	\end{UCtrayectoria}

%--------------------------------------		
		\begin{UCtrayectoriaA}{A}{El cliente olvide su membresía}
			\UCpaso[\UCactor] Captura el correo electrónico del cliente y su nombre completo.% {\bf MSG1-}``El Estudiante [{\em Número de Boleta}] aun no puede inscribirse al seminario.''.
			\UCpaso[\UCactor] Confirma la operación oprimiendo el botón de buscar.% \IUbutton{Aceptar}.
			\UCpaso Continua en el paso \ref{CU15ValidarQR} del \UCref{CU15}.
		\end{UCtrayectoriaA}
		
%--------------------------------------
		\begin{UCtrayectoriaA}{B}{El cliente tiene menos de una semana sin pagar}
			\UCpaso Muestra mensaje de Realizar pronto pago, permitir entrada.
			\UCpaso[\UCactor] Oprime el botón de aceptar.% \IUbutton{Salir}.
			\UCpaso Muestra mensaje: ¿Quiere realizar el pago?.
			\UCpaso[\UCactor] Toma una desición.
			\UCpaso[] termina caso de uso.
		\end{UCtrayectoriaA}

%--------------------------------------
		\begin{UCtrayectoriaA}{C}{El cliente tenga mas de una semana sin pagar}
			\UCpaso Muestra mensaje de no permitir entrada.%{\bf MSG2-}``El Estudiante [{ Número de Boleta}] no cumple con los requisitos para inscribirse al Seminario [{\em Nombre del Seminario seleccionado}].''.
			\UCpaso[\UCactor] Continua con el paso \ref{CU15RealizarPago} de la trayectoria B.
		\end{UCtrayectoriaA}

%--------------------------------------
% Puntos de extensión
%\subsection{Puntos de extensión}
%\UCExtenssionPoint{
	% Cuando:
%	Desea conocer las materias cursadas.
%}{
	% Durante la región:
%	Del paso 4 al paso 9.
%}{
	% Casos de uso a los que extiende:
%	\hyperlink{CU3.4}{CU3.4 Consultar historial académico}.
%}
		
		
		
%-------------------------------------- TERMINA descripción del caso de uso.


%-------------------------------------- COMIENZA descripción del caso de uso.

%\begin{UseCase}[archivo de imágen]{UCX}{Nombre del Caso de uso}{
%--------------------------------------
	\begin{UseCase}{CU23}{Inscribir actividad}{
		Permite  al cliente poder inscribirse en una actividad que se imparte en una sucursal.
	}
		\UCitem{Versión}{\color{Gray}0.1}
		\UCitem{Autor}{\color{Gray}Daniel}
		\UCitem{Supervisa}{\color{Gray}Luis}
		\UCitem{Actor}{\hyperlink{Cliente}{Cliente}}
		\UCitem{Propósito}{Que el usuario pueda ir a la actividad sin que le nieguen el acceso y conozca el horario en el que debe asistir.}
		\UCitem{Entradas}{Nombre Actividad, Sucursal, horario}
		\UCitem{Origen}{Teclado}
		\UCitem{Salidas}{Mensaje,Nombre Actividad,Sucursal,horario}
		\UCitem{Destino}{Pantalla del cliente}
		\UCitem{Precondiciones}{ Que el cliente haya iniciado sesión, que exista la actividades que se desea inscribir, que haya cupo en la actividad deseada.}
		\UCitem{Postcondiciones}{El cliente vera la actividad inscrita en su lista de actividades.}
		\UCitem{Errores}{No haya cupo en la actividad por inscribir}
		\UCitem{Observaciones}{}
	\end{UseCase}
%--------------------------------------
	\begin{UCtrayectoria}{Principal}
		\UCpaso[\UCactor] Desde la pantalla donde termina el CU22 seleccionar el botón de inscribir actividad.
		\UCpaso Muestra la pantalla de inscribir actividad.
		\UCpaso Muestra los campos que se deberán llenar para inscribir la actividad.
		\UCpaso[\UCactor] llena los campos y dar clic en inscribir.
		\UCpaso[\UCactor] confirma la operación oprimiendo el botón de aceptar. 
		\UCpaso Valida que la actividad y el horario que el usuario eligió exista.
		\UCpaso Valida que haya cupo en la actividad que eligió el usuario.
		\UCpaso Despliega msg3 "Actividad inscrita".
		\UCpaso Muestra la pantalla de inicio de cliente.	
	\end{UCtrayectoria}
%-------------------------------------- TERMINA descripción del caso de uso.


%-------------------------------------- COMIENZA descripción del caso de uso.

%\begin{UseCase}[archivo de imágen]{UCX}{Nombre del Caso de uso}{
%--------------------------------------
\begin{UseCase}{CU25}{Calificar Actividad.}{
		El cliente selecciona una puntuación para la actividad en un intervalo del 1 al 5 y se quedara guardada.
	}
	\UCitem{Versión}{\color{Gray}0.1}
	\UCitem{Autor}{\color{Gray}Jazmin Camarillo Martinez}
	\UCitem{Supervisa}{\color{Gray}}
	\UCitem{Actor}{\hyperlink{Cliente}{Cliente}}
	\UCitem{Propósito}{ Que el cliente pueda transmitir una retro alimentación para el gimnasio y que pueda tener un mejor servicio. }
	\UCitem{Entradas}{Nombre Actividad, Calificación, Comentario }
	\UCitem{Origen}{Mouse y teclado.}
	\UCitem{Salidas}{Mensaje, Calificación}
	\UCitem{Destino}{Pantalla del cliente}
	\UCitem{Precondiciones}{ Que el cliente haya iniciado sesión, que este inscrito a alguna actividad.}
	\UCitem{Postcondiciones}{Que el cliente no este inscrito en la actividad que quiere calificar,que no exista la actividad que quiere calificar el cliente, que el cliente califique fuera del rango permitido.}
	\UCitem{Errores}{}
	\UCitem{Observaciones}{}
\end{UseCase}
%--------------------------------------

\begin{UCtrayectoria}{Principal}
	\UCpaso[\UCactor] Desde la pantalla donde termina el CU22 seleccionar el botón de Actividades.
	\UCpaso[\UCactor] Selecciona el botón de calificar actividad.
	\UCpaso Despliega los campos a llenar para calificar una actividad en la misma pantalla.
	\UCpaso[\UCactor] Selecciona una actividad
	\UCpaso[\UCactor] Asigna una calificación y un comentario que sera de manera opcional.
	\UCpaso[\UCactor] Confirma la operación oprimiendo el botón de "aceptar"..
	\UCpaso Valida que la actividad que seleccionó el usuario exista y que este inscrito en esa actividad.
	\UCpaso Mustra MSG4 "Calificación hecha" y la calificación de la actividad.
	
\end{UCtrayectoria}

\begin{UCtrayectoriaA}{A}{No haya cupo en Actividad}
	\UCpaso Captura la actividad que el cliente quiere.
	\UCpaso[\UCactor] Manda mensaje de sin cupo.
	\UCpaso Confirma de enterado oprimiendo el botón de aceptar.
	\end{UCtrayectoriaA}
%-------------------------------------- TERMINA descripción del caso de uso.



%\begin{figure}[htbp]
%	\begin{center}
%		\includegraphics[width=.8\textwidth]{images/vistas/buscarcliente}
%		\caption{Buscar cliente}
%		\label{fig:Ventas}
%	\end{center}
%\end{figure}
