% \IUref{IUAdmPS}{Administrar Planta de Selección}
% \IUref{IUModPS}{Modificar Planta de Selección}
% \IUref{IUEliPS}{Eliminar Planta de Selección}

% 


% Copie este bloque por cada caso de uso:
%-------------------------------------- COMIENZA descripción del caso de uso.

%\begin{UseCase}[archivo de imágen]{UCX}{Nombre del Caso de uso}{
%--------------------------------------
	\begin{UseCase}{CU1}{Iniciar sesión de jefe de inmobiliario}{
	
		Validar al jefe de inmobiliario para que pueda tener acceso a la información de la sucursal.
	}
		\UCitem{Versión}{\color{Gray}0.4}
		\UCitem{Autor}{\color{Gray}Jazmin Camarillo Martínez}
		\UCitem{Supervisa}{\color{Gray}Francisco}
		\UCitem{Actor}{Jefe de inmobiliario}
		\UCitem{Propósito}{Para que el jefe de inmobiliario sea el único que pueda realizar acciones como eliminar modificar o agregar actividades, sucursales y áreas, y mantener la información actualizada.}
		\UCitem{Entradas}{Nombre de usurario y password.}
		\UCitem{Origen}{Teclado}
		\UCitem{Salidas}{mensaje de bienvenido, nombre completo del jefe de inmobiliario, cargo del J.I. nombre de la sucursal.}
		\UCitem{Destino}{Pantalla del jefe de inmobiliario.}
		\UCitem{Precondiciones}{El jefe de inmobiliario debe estar registrado, también debe tener una cuenta correcta para poder acceder como jefe de inmobiliario y que no haya una sesión iniciada.}
		\UCitem{Postcondiciones}{El jefe de inmobiliario deberá ver un menú con; perfil, sucursales, actividades y áreas, y sus datos de salida. }
		\UCitem{Errores}{}
		\UCitem{Tipo}{Caso de uso primario}
		\UCitem{Observaciones}{}
	\end{UseCase}
%--------------------------------------
	\begin{UCtrayectoria}{Principal}
		\UCpaso[\UCactor] Introduce su Nombre de usuario y su password para poder ingresar vía la  \IUref{IU1}{Pantalla de Inicio de Sesión del Jefe de Inmobiliario.}\label{CU1LoginJI}.
		\UCpaso[\UCactor] Confirma la operación presionando el botón Iniciar Sesión.
		\UCpaso Verifica que el nombre y password son correctos para ingresar sesión del jefe de inmobiliario.
		\UCpaso Despliega la \IUref{IU2}{Pantalla inicio del jefe de inmobiliario}.
	\end{UCtrayectoria}

%--------------------------------------		
		\begin{UCtrayectoriaA}{A}{El nombre de sesión no existe}
			\UCpaso[\UCactor] Muestra el Mensaje {\bf MSG1-}``El empleado [{\em Nombre de sesión}] no existe.''.
			\UCpaso[\UCactor] Introduce nombre usuario correcto.
			\UCpaso[] Continua con el paso 3 del \UCref{CU1}.
		\end{UCtrayectoriaA}
		
%--------------------------------------
		\begin{UCtrayectoriaA}{B}{Contraseña incorrecta}
			\UCpaso Muestra el Mensaje {\bf MSG1¿2-}``El password no es correcto.''.
			\UCpaso[\UCactor] Introduce password correcto.
			\UCpaso[] Continua con el paso 3 del \UCref{CU1}.
		\end{UCtrayectoriaA}


%--------------------------------------
% Puntos de extensión
\subsection{Puntos de extensión}
\UCExtenssionPoint{
	% Cuando:
	Desea acceder a las sucursales.
	Desea acceder a las actividades.
	Desea acceder a las áreas.
	Desea acceder a su perfil.
}{
	% Durante la región:
	En el paso 5.
}{
	% Casos de uso a los que extiende:
	\hyperlink{CU2}{CU2 Consultar Actividades de sucursal}.
}
		
		
% Copie este bloque por cada caso de uso:
%-------------------------------------- COMIENZA descripción del caso de uso.

%\begin{UseCase}[archivo de imágen]{UCX}{Nombre del Caso de uso}{
%--------------------------------------
\begin{UseCase}{CU2}{Consultar Actividades de sucursal}{
		Acceder al sistema a través del inicio de sesión del jefe de inmobiliario y ver datos generales de las actividades que hay en la sucursal.
	}
	\UCitem{Versión}{\color{Gray}0.4}
	\UCitem{Autor}{\color{Gray}Jazmin Camarillo Martínez}
	\UCitem{Supervisa}{\color{Gray}Francisco}
	\UCitem{Actor}{Jefe de inmobiliario}
	\UCitem{Propósito}{Para que el jefe pueda ver las actividades que tiene la sucursal y saber si sus datos son correctos y/o poderlos modificar.}
	\UCitem{Entradas}{Botón de actividad.}
	\UCitem{Origen}{Mouse}
	\UCitem{Salidas}{Nombre de la sucursal, nombre del isntructor, hora y dia dde la actividad.}
	\UCitem{Destino}{Pantalla del jefe de inmobiliario.}
	\UCitem{Precondiciones}{El jefe de inmobiliario debe inciiar sesion para ver las actividades, debe haber al menos una actividad.}
	\UCitem{Postcondiciones}{}
	\UCitem{Errores}{}
	\UCitem{Tipo}{Caso de uso primario}
	\UCitem{Observaciones}{}
\end{UseCase}
%--------------------------------------
\begin{UCtrayectoria}{Principal}
	\UCpaso[\UCactor] Introduce su Nombre de usuario y su password para poder ingresar vía la  \IUref{IU1}{Pantalla de Inicio de Sesión del Jefe de Inmobiliario.}\label{CU1LoginJI}.
	\UCpaso[\UCactor] Confirma la operación presionando el botón Iniciar Sesión.
	\UCpaso Verifica que el nombre y password son correctos para ingresar sesión del jefe de inmobiliario.
	\UCpaso Despliega la \IUref{IU2}{Pantalla inicio del jefe de inmobiliario}.
	\UCpaso[\UCactor] Selecciona en su menú el botón de las actividades. 
	\UCpaso Despliega la \IUref{IU3}{Pantalla Actividades}.
	
\end{UCtrayectoria}


%--------------------------------------
% Puntos de extensión
\subsection{Puntos de extensión}
\UCExtenssionPoint{
	% Cuando:
	Desee ver información completa de la actividad.
	Desee editar la información.
	Desee agregar una nueva actividad.
}{
	% Durante la región:
	En el paso 5.
}{
	% Casos de uso a los que extiende:
	\hyperlink{CU3}{CU3 Modificar actividad de sucursal}.
	\hyperlink{CU4}{CU4 Ver actividad de sucursal}.
	\hyperlink{CU5}{CU5 Agregar actividad de sucursal}.
	\hyperlink{CU6}{CU6 Buscar Actividad en específico}.
}		
		


%\begin{UseCase}[archivo de imágen]{UCX}{Nombre del Caso de uso}{
%--------------------------------------
\begin{UseCase}{CU3}{Modificar actividad de sucursal}{
	Modificar los datos permitidos a través de un formulario que despliega el sistema.
	}
	\UCitem{Versión}{\color{Gray}0.4}
	\UCitem{Autor}{\color{Gray}Jazmin Camarillo Martínez}
	\UCitem{Supervisa}{\color{Gray}Francisco}
	\UCitem{Actor}{Jefe de inmobiliario}
	\UCitem{Propósito}{Para que el jefe de inmobiliario pueda mantener la informacion correcta de las actividades que hay en el sistema.}
	\UCitem{Entradas}{Nombre de la actividad, el dia y la hora que se imparte la actividad, nombre del instructor,imagen(es) y descripcion.}
	\UCitem{Origen}{Mouse y teclado}
	\UCitem{Salidas}{Numero de actividad, nombre de la actividad, nombre de la sucursal, el dia y la hora que se imparte la actividad,nombre del instructor, la descripcion de la actividad, dirección de la sucursal, imagen(es) y mensaje de operación exitosa.}
	\UCitem{Destino}{Pantalla del jefe de inmobiliario.}
	\UCitem{Precondiciones}{La actividad debe estar registrada para poder ser modificada. Debe existir en el sistema el nombre del isntructor.}
	\UCitem{Postcondiciones}{La actividad quedará modificada con los nuevos datos, si es correcto el formato con el que se lleno el formulario, la información se verá actualizada en la consulta de ver actividad.}
	\UCitem{Errores}{Formulario no este llenado de acuerdo al formato.}
	\UCitem{Tipo}{Caso de uso primario}
	\UCitem{Observaciones}{Todos los campos del formulario son obligatorios.}
\end{UseCase}
%--------------------------------------
\begin{UCtrayectoria}{Principal}
	\UCpaso[\UCactor] Selecciona en su menú el botón de las actividades vía \IUref{IU3}{Pantalla Actividades}.
	\UCpaso Despliega la \IUref{IU3}{Pantalla Actividades}.
	\UCpaso[\UCactor] Selecciona botón de Editar que aparece en la \IUref{IU3}{Pantalla Actividades}.
	\UCpaso Despliega la \IUref{IU4}{Pantalla Editar Actividad}.
	\UCpaso[\UCactor] Selecciona el recuadro que quiera modificar .
	\UCpaso[\UCactor] Borra los datos del recuadro que seleccionó para poner el nuevo dato.
	\UCpaso[\UCactor] Pulsa el botón de aceptar para confirmar la operación que se realizó.
	\UCpaso Guarda los datos que están en los recuadros.	
	\UCpaso Despliega la \IUref{IU5}{Pantalla Edición terminada}.
	\UCpaso[\UCactor] Verifica que los datos estén como el desea y oprime el botón aceptar.
	\UCpaso Despliega la \IUref{IU3}{Pantalla Actividades}.
\end{UCtrayectoria}

%\BRref{BR129}{Determinar si un Estudiante puede inscribir Seminario.} \Trayref{A}
%--------------------------------------		
\begin{UCtrayectoriaA}{A}{El jefe de inmobiliario deja información en blanco}
	\UCpaso Muestra el Mensaje {\bf MSG3-}``Todos los campos son obligatorios.''.
	\UCpaso[\UCactor] Llena los campos faltantes.
	\UCpaso[\UCactor] Continua con el paso 8 del \UCref{CU3}.
\end{UCtrayectoriaA}


% Copie este bloque por cada caso de uso:
%-------------------------------------- COMIENZA descripción del caso de uso.

%\begin{UseCase}[archivo de imágen]{UCX}{Nombre del Caso de uso}{
%--------------------------------------
\begin{UseCase}{CU4}{Ver actividad de sucursal}{
		El sistema muestrará la información completa de una actividad seleccionada.
	}
	\UCitem{Versión}{\color{Gray}0.4}
	\UCitem{Autor}{\color{Gray}Jazmin Camarillo Martínez}
	\UCitem{Supervisa}{\color{Gray}Francisco}
	\UCitem{Actor}{Jefe de inmobiliario}
	\UCitem{Propósito}{Para que el jefe de inmobiliario pueda revisar que los datos de la actividad sean correctos.}
	\UCitem{Entradas}{botón de ver actividad.}
	\UCitem{Origen}{Teclado}
	\UCitem{Salidas}{Numero de actividad, nombre de la actividad, nombre de la sucursal, el dia y la hora que se imparte la actividad,nombre del instructor, la descripcion de la actividad, dirección de la sucursal, imagen(es).}
	\UCitem{Destino}{Pantalla del jefe de inmobiliario.}
	\UCitem{Precondiciones}{El jefe de inmobiliario debe estar dentro de susesión para poder ver la información. La actividad que desee ver, debe estar registrada.}
	\UCitem{Postcondiciones}{}
	\UCitem{Errores}{Actividad inexistente}
	\UCitem{Tipo}{Caso de uso primario}
	\UCitem{Observaciones}{}
\end{UseCase}
%--------------------------------------
\begin{UCtrayectoria}{Principal}
	\UCpaso[\UCactor] Ingresa a la \IUref{IU3}{Pantalla de Actividades.}\label{CU1LoginJI}.
	\UCpaso[\UCactor] Oprime el botón de ver.
	\UCpaso Despliega la \IUref{IU6}{Pantalla Ver Actividad}.
	\UCpaso Oprime el botón de aceptar para regresar a la  \IUref{IU3}{Pantalla de Actividades.}\label{CU1LoginJI}.
\end{UCtrayectoria}


%--------------------------------------
% Puntos de extensión
\subsection{Puntos de extensión}
\UCExtenssionPoint{
	% Cuando:
	Desea modificar datos de actividad.
}{
	% Durante la región:
	En el paso 3.
}{
	% Casos de uso a los que extiende:
	\hyperlink{CU3}{CU3 Modificar actividad de sucursal}.
}



% Copie este bloque por cada caso de uso:
%-------------------------------------- COMIENZA descripción del caso de uso.

%\begin{UseCase}[archivo de imágen]{UCX}{Nombre del Caso de uso}{
%--------------------------------------
\begin{UseCase}{CU5}{Agregar actividad a la sucursal}{
		El Jefe de inmobiliario podrá agregar una nueva actividad al sistema.
	}
	\UCitem{Versión}{\color{Gray}0.4}
	\UCitem{Autor}{\color{Gray}Jazmin Camarillo Martínez}
	\UCitem{Supervisa}{\color{Gray}Francisco}
	\UCitem{Actor}{Jefe de inmobiliario}
	\UCitem{Propósito}{Para que cada que se agregue una nueva actividad a la sucursal se pueda incorporar su información al sistema.}
	\UCitem{Entradas}{}
	\UCitem{Origen}{Teclado}
	\UCitem{Salidas}{.}
	\UCitem{Destino}{Pantalla del jefe de inmobiliario.}
	\UCitem{Precondiciones}{.}
	\UCitem{Postcondiciones}{El jefe de inmobiliario deberá ver un menú con; perfil, sucursales, actividades y áreas, y sus datos de salida. }
	\UCitem{Errores}{}
	\UCitem{Tipo}{Caso de uso primario}
	\UCitem{Observaciones}{}
\end{UseCase}
%--------------------------------------
\begin{UCtrayectoria}{Principal}
	\UCpaso[\UCactor] Introduce su Nombre de usuario y su password para poder ingresar vía la  \IUref{IU1}{Pantalla de Inicio de Sesión del Jefe de Inmobiliario.}\label{CU1LoginJI}.
	\UCpaso[\UCactor] Confirma la operación presionando el botón Iniciar Sesión.
	\UCpaso Verifica que el nombre y password son correctos para ingresar sesión del jefe de inmobiliario.
	\UCpaso Despliega la \IUref{IU2}{Pantalla inicio del jefe de inmobiliario}.
\end{UCtrayectoria}

%--------------------------------------		
\begin{UCtrayectoriaA}{A}{El nombre de sesión no existe}
	\UCpaso[\UCactor] Muestra el Mensaje {\bf MSG1-}``El empleado [{\em Nombre de sesión}] no existe.''.
	\UCpaso[\UCactor] Introduce nombre usuario correcto.
	\UCpaso[] Continua con el paso 3 del \UCref{CU1}.
\end{UCtrayectoriaA}

%--------------------------------------
\begin{UCtrayectoriaA}{B}{Contraseña incorrecta}
	\UCpaso Muestra el Mensaje {\bf MSG12-}``El password no es correcto.''.
	\UCpaso[\UCactor] Introduce password correcto.
	\UCpaso[] Continua con el paso 3 del \UCref{CU1}.
\end{UCtrayectoriaA}


%--------------------------------------
% Puntos de extensión
\subsection{Puntos de extensión}
\UCExtenssionPoint{
	% Cuando:
	Desea acceder a las sucursales.
	Desea acceder a las actividades.
	Desea acceder a las áreas.
	Desea acceder a su perfil.
}{
	% Durante la región:
	En el paso 5.
}{
	% Casos de uso a los que extiende:
	\hyperlink{CU2}{CU2 Consultar Actividades de sucursal}.
}
