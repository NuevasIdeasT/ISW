% \IUref{IUAdmPS}{Administrar Planta de Selección}
% \IUref{IUModPS}{Modificar Planta de Selección}
% \IUref{IUEliPS}{Eliminar Planta de Selección}

% 

%\begin{UseCase}[archivo de imágen]{UCX}{Nombre del Caso de uso}{
%--------------------------------------

\begin{itemize}
\item \textbf{ID}: CP 35
\item \textbf{ALCANCE}: CU35
\item \textbf{NOMBRE}: Registrar Instructor
\item \textbf{PREPARACIÓN}:\\
		-El Gerente tiene que tener una sesión iniciada.
		-Debe existir al menos una Actividad Registradaen la DB.
\begin{itemize}
\item \textbf{CP 1}:
\item \textbf{DATOS DE ENTRADA}:\\
-Luis Quiroz\\
-Record 56, Col. portales,CDMX.\\
-LuisQ\\
-12345678\\
	\begin{center}			
	\begin{tabular}{||l|l|l|l||}
		\hline
		\hline
			CP 1\\
			PASOS\\
			\hline 1.- Se selecciona el boton ''Agregar Instructor''\\
			\hline 2.- Se instroduce ''Luis Quiroz'' en el campo  Nombre Completo.\\
			\hline 3.- Se instroduce ''Record 56, Col. portales,CDMX.'' en el campo  Dirección.\\
			\hline 4-. Se introduce  ''LuisQ'' en el campo Nombre de Usuario.\\
			\hline 5-. Se introduce  12345678  en el campo Password.\\
            \hline 6.-Se da clic en el botón ''Nueva Actividad''\\
            \hline 7-.Se selecciona la actividad ''Pesas''\\
            \hline 8.-Se da clic en el botón ''Guardar''.\\
        \hline
		\hline
	\end{tabular}
	\end{center}
\item \textbf{SALIDAS}: Redirección a la pantalla \IUref{IU3}{Editar Instructor} y aparece la leyenda \textbf{Registro Exitoso}.
\end{itemize}
\begin{itemize}
\item \textbf{CP 2}:
\item \textbf{DATOS DE ENTRADA}:\\
-LuisQU\\
-12345678\\
	\begin{center}			
	\begin{tabular}{||l|l|l|l||}
		\hline
		\hline
			CP 2\\
			PASOS\\
			\hline 1.- Se selecciona el boton ''Agregar Instructor''\\
			\hline 4-. Se introduce  ''LuisQU'' en el campo Nombre de Usuario.\\
			\hline 5-. Se introduce  12345678  en el campo Password.\\
            \hline 6.-Se da clic en el botón ''Nueva Actividad''\\
            \hline 7-.Se selecciona la actividad ''Pesas''\\
            \hline 8.-Se da clic en el botón ''Guardar''.\\
        \hline
		\hline
	\end{tabular}
	\end{center}
\item \textbf{SALIDAS}:Aparece en el campo 'Nombre Completo' la leyenda \textbf{Nombre Invalido} y en el campo 'Dirección' la leyenda \textbf{Dirección Invalida} .
\end{itemize}
\end{itemize}
%------------------------------------------------------

\begin{itemize}
\item \textbf{ID}: CP 36
\item \textbf{ALCANCE}: CU 36
\item \textbf{NOMBRE}: Modificar Instructor
\item \textbf{PREPARACIÓN}:\\
		-El Gerente tiene que tener una sesión iniciada.
		-Debe existir al menos una Actividad Registrada en la DB.
		-Debe existir al menos un Instructor Registrada en la DB.
\begin{itemize}
\item \textbf{CP 1}:
\item \textbf{DATOS DE ENTRADA}:\\
-Selección Instructor "Daniel Romero"
-Daniel Romero Lopez
	\begin{center}			
	\begin{tabular}{||l|l|l|l||}
		\hline
		\hline
			CP 1\\
			PASOS\\
			\hline 1.- Se selecciona el boton de edición\\
			\hline 2.- Se despliega la pantalla con los datos del instructror.\\
			\hline 3.- Se introduce en el campo de nombre la cadena ''Daniel Romero Lopez''.\\
			\hline 4-. Se selecciona la actividad ''Pesas''\\
			\hline 5-. Se selecciona el horario '''8:00-9:00''\\
            \hline 6.-Se da clic en el botón ''Guardar''\\
        \hline
		\hline
	\end{tabular}
	\end{center}
\item \textbf{SALIDAS}: Redirección a la pantalla \IUref{IU3}{Instructor} y aparece la leyenda ''Daniel Romero Lopez ha actualizado con éxito''
\end{itemize}
\begin{itemize}
\item \textbf{CP 2}:
\item \textbf{DATOS DE ENTRADA}:\\
-Selección Instructor "Daniel Romero"
-Daniel Romero Lopez
	\begin{center}			
	\begin{tabular}{||l|l|l|l||}
		\hline
		\hline
			CP 2\\
			PASOS\\
			\hline 1.- Se selecciona el boton de edición\\
			\hline 2.- Se despliega la pantalla con los datos del instructror.\\
			\hline 3.- Se introduce en el campo de nombre la cadena ''Daniel Romero Lopez''.\\
			\hline 4-. Se borra el campo 'Dirección'\\
			\hline 4-. Se selecciona la actividad ''Pesas''\\
			\hline 5-. Se selecciona el horario '''8:00-9:00''\\
            \hline 6.-Se da clic en el botón ''Guardar''\\
        \hline
		\hline
	\end{tabular}
	\end{center}
\item \textbf{SALIDAS}:Aparece en el campo 'Dirección' la leyenda \textbf{Campo Requerido}.
\end{itemize}
\end{itemize}
%------------------------------------------------------

\begin{itemize}
\item \textbf{ID}: CP 37
\item \textbf{ALCANCE}: CU 37
\item \textbf{NOMBRE}: Eliminar Instructor
\item \textbf{PREPARACIÓN}:\\
		-El Gerente tiene que tener una sesión iniciada.\\
		-Debe existir al menos un Instructor Registrada en la DB.
\begin{itemize}
\item \textbf{CP 1}:
\item \textbf{DATOS DE ENTRADA}:\\
-Selección Instructor "Daniel Romero"
	\begin{center}			
	\begin{tabular}{||l|l|l|l||}
		\hline
		\hline
			CP 1\\
			PASOS\\
			\hline 1.- Se selecciona el boton de ''eliminar''\\
			\hline 2.- Se despliega una alerta con la leyenda ''seguro que deas eliminar al instructor''\\ y las opciones ''Aceptar'' y ''Cancelar''.\\
			\hline 3.- Se selcciona la opción ''Aceptar''\\
        \hline
		\hline
	\end{tabular}
	\end{center}
\item \textbf{SALIDAS}: Redirección a la pantalla \IUref{IU3}{Instructor} y aparece la leyenda ''Borrado exitoso'' y ya no se aprecia en la tabla al instructor borrado.
\end{itemize}
\begin{itemize}
\item \textbf{CP 2}:
\item \textbf{DATOS DE ENTRADA}:\\
-Selección Instructor "Daniel Romero"
	\begin{center}			
	\begin{tabular}{||l|l|l|l||}
		\hline
		\hline
			CP 2\\
			PASOS\\
			\hline 1.- Se selecciona el boton de ''eliminar''\\
			\hline 2.- Se despliega una alerta con la leyenda ''seguro que deas eliminar al instructor'\\' y las opciones ''Aceptar'' y ''Cancelar''.\\
			\hline 3.- Se selcciona la opción ''Cancelar''\\
        \hline
		\hline
	\end{tabular}
	\end{center}
\item \textbf{SALIDAS}:Se muestra la pantalla \IUref{IU3}{Instructor} sin  ningún cambio.
\end{itemize}
\end{itemize}
%------------------------------------------------------

\begin{itemize}
\item \textbf{ID}: CP 38
\item \textbf{ALCANCE}: CU 38
\item \textbf{NOMBRE}: Asignar Actividad al Instructor
\item \textbf{PREPARACIÓN}:\\
		-El Gerente tiene que tener una sesión iniciada.\\
		-Debe existir al menos un Instructor Registrada en la DB.
		-Debe existir al menos una Acttividad Registrada en la DB.
		-Se debe haber seleccionado una Actividad.
\begin{itemize}
\item \textbf{CP 1}:
\item \textbf{DATOS DE ENTRADA}:\\
	\begin{center}			
	\begin{tabular}{||l|l|l|l||}
		\hline
		\hline
			CP 1\\
			PASOS\\
			\hline 1.- Se Despliegan los horarios disponibles\\
			\hline 2.- Se selecciona el horario 8:00-9:00\\
        \hline
		\hline
	\end{tabular}
	\end{center}
\item \textbf{SALIDAS}: Se aprecia en la pantalla que se asigno el horario seleccionado.
\end{itemize}
\begin{itemize}
\item \textbf{CP 2}:
\item \textbf{DATOS DE ENTRADA}:\\
-Selección Instructor "Daniel Romero"
	\begin{center}			
	\begin{tabular}{||l|l|l|l||}
		\hline
		\hline
			CP 2\\
			PASOS\\
			\hline 1.- Se Despliegan los horarios disponibles\\
        \hline
		\hline
	\end{tabular}
	\end{center}
\item \textbf{SALIDAS}:Se muestra la pantalla \IUref{IU3}{Instructor} la leyenda \textbf{Se debe seleccionar al menos un horario}.
\end{itemize}
\end{itemize}
%------------------------------------------------------

