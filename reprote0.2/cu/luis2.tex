% \IUref{IUAdmPS}{Administrar Planta de Selección}
% \IUref{IUModPS}{Modificar Planta de Selección}
% \IUref{IUEliPS}{Eliminar Planta de Selección}

% 

%\begin{UseCase}[archivo de imágen]{UCX}{Nombre del Caso de uso}{
%--------------------------------------
\begin{UseCase}{CU34}{Registrar Instructor}{
		Crear un nuevo instructor en el sistema introduciendo sus datos personales y asignándole al menos una actividad.
	}
	\UCitem{Versión}{\color{Gray}0.4}
	\UCitem{Autor}{\color{Gray}Quiroz Olmedo Luis Eduardo}
	\UCitem{Supervisa}{\color{Gray}Francisco}
	\UCitem{Actor}{Gerente}
	\UCitem{Propósito}{Permitir al Gerente registrar nuevos Instructores en el sistema para que se les pueda asignar actividades nuevas o existentes y poder gestionar a los instructores.}
	\UCitem{Entradas}{•Nombre\newline
	•Dirección\newline
	•Nombre de Usuario\newline
	•Password}
	\UCitem{Origen}{Mouse y Teclado}
	\UCitem{Salidas}{Mensaje de confirmación.}
	\UCitem{Destino}{Pantalla de Instructores.}
	\UCitem{Precondiciones}{El jefe de inmobiliario debe inciiar sesion para ver las actividades, debe haber al menos una actividad.}
	\UCitem{Postcondiciones}{•Que el actor tenga una sesión iniciada
	\newline•Que el nombre de Usuario no este dado de alta en el sistema}
	\UCitem{Errores}{•Campos de Texto nulos o incompletos.\textbf{Trayectoria Alternativa A y B} \newline
	•Todas las actividades ocupadas.\textbf{Trayectoria Alternativa C}\newline
	•El actor no selecciona ninguna actividad.\textbf{Trayectoria Alternativa D}\newline}
	\UCitem{Tipo}{Caso de uso primario}
	\UCitem{Observaciones}{Todos los campos del formulario son obligatorios.}
\end{UseCase}
%--------------------------------------
\begin{UCtrayectoria}{Este caso de uso inicia cuando un actor se encuentra en la pantalla de \IUref{IU35}{Pantalla de Instructores}}
	\UCpaso[\UCactor] selecciona el boton ''Agregar Nuevo Instructor''.
	\UCpaso despliega la pantalla \IUref{IU351}{Pantalla Nuevo Instructor}
	\UCpaso[\UCactor]introduce el nombre completo del instructor en el campo de texto “Nombre completo”.
	\UCpaso sistema valida que los valores introducidos cumplen con las caracteristicas de un nombre.[aA-zA]
	\UCpaso[\UCactor]introduce la dirección del instructor en el campo de texto “Dirección”.
	\UCpaso sistema valida que los valores introducidos cumplen con las caracteristicas de una dirección.[aA-zA1-9 '.' '-']
	\UCpaso[\UCactor]introduce un nombre de usuario del instructor en el campo de texto “Nombre de Usuario”.
	\UCpaso sistema valida que los valores introducidos cumplen con las caracteristicas de un nombre [aA-zA0-9] y que el nombre de usuario no exista.
	\UCpaso[\UCactor]introduce un password del instructor en el campo de texto “Password”.
	\UCpaso sistema valida que los valores introducidos cumplen con las caracteristicas de una password.[aA-zA1-9]{8}
	\UCpaso selecciona el boton "Nueva Actividad"
	\UCpaso consulta que actividades tienen horarios libres y las despliega en la pantalla en forma de lista.
	\UCpaso[\UCactor] selecciona una actividad.
	\UCpaso \IUref {CU38}
	\UCpaso [\UCactor] da clic en el icono guardar.
	\UCpaso valida que al menos una actividad ha sido asignada.
	\UCpaso crea una nueva entidad Instructor con los datos introducidos ligado con las o la actividad seleccionadas y sus horarios seleccionados.
	\UCpaso despliega la pantalla \IUref{IU35}{Pantalla de Instructores} con el mensaje {\bf MSG7-} “Registro Exitoso”.
\end{UCtrayectoria}
%--------------------------------------
\begin{UCtrayectoriaA}{A}{El campo ''Nombre completo'' tiene menos de dos palabras o es nulo.}
			\UCpaso despliega un mensaje dentro del campo con la leyenda “Nombre invalido” y limpia el campo “Nombre Completo” y se continúa en el paso [1].
\end{UCtrayectoriaA}
%--------------------------------------
\begin{UCtrayectoriaA}{B}{El campo ''Direción'' tiene menos de dos palabras o es nulo.}
			\UCpaso despliega un mensaje dentro del campo con la leyenda “Dirección invalida” y limpia el campo “Direccion” y se continúa en el paso [1].
\end{UCtrayectoriaA}
%--------------------------------------
\begin{UCtrayectoriaA}{C}{Todas las Actividades se encuentran ocupadas}
		   \UCpaso despliega un mensaje en pantalla con la leyenda {\bf MSG8-} “No se cuenta con Actividades disponibles para asignar”.
		   \UCpaso despliega la pantalla \IUref{IU35}{Pantalla de Instructores} cancelando el registro y no guardando ningún dato.
\end{UCtrayectoriaA}
%--------------------------------------
\begin{UCtrayectoriaA}{D}{El actor no selecciono ninguna actividad}
		   \UCpaso despliega un mensaje en pantalla con la leyenda {\bf MSG9-} “Se debe seleccionar al menos una actividad” y se continua en el paso [9].	  
\end{UCtrayectoriaA}
%--------------------------------------
\begin{UCtrayectoriaA}{E}{El actor selecciona el boton ''Cancelar''}
		   \UCpaso muestra una alerta en pantalla con la leyenda {\bf MSG10-} “Estas seguro que deseas Cancelar” y dando las opciones Aceptar y Cancelar.
		   \UCpaso[\UCactor] selecciona ''Aceptar''.
		   \UCpaso despliega la pantalla \IUref{IU35}{Pantalla de Instructores}cancelando el registro y no guardando ningún dato.
\end{UCtrayectoriaA}

%--------------------------------------

%-------------------------------------- COMIENZA descripción del caso de uso.

%\begin{UseCase}[archivo de imágen]{UCX}{Nombre del Caso de uso}{
%--------------------------------------
\begin{UseCase}{CU35}{Listar Instructores}{
		El sistema genera una lista de todos los instructores que se encuentran activos.
	}
	\UCitem{Versión}{\color{Gray}0.4}
	\UCitem{Autor}{\color{Gray}Quiroz Olmedo Luis}
	\UCitem{Supervisa}{\color{Gray}Francisco}
	\UCitem{Actor}{Sistema}
	\UCitem{Propósito}{Listar solo los Instructores que no se encuentran declarados con bandera de borrado y mostrarlos para su edición.}
	\UCitem{Entradas}{}
	\UCitem{Origen}{}
	\UCitem{Salidas}{Lista de entidad Insctructor}
	\UCitem{Destino}{Pantalla de Instructores}
	\UCitem{Precondiciones}{• Que almenos exista un Instructor registrado}
	\UCitem{Postcondiciones}{Se tendrá la lista de todos los Instructores registrados}
	\UCitem{Errores}{•Todos los instructores se encuentran bloqueados o no hay instructores registrados.\textbf{Trayectoria Alternativa A}}
	\UCitem{Tipo}{Caso de uso secundario}
	\UCitem{Observaciones}{}
\end{UseCase}
%--------------------------------------
\begin{UCtrayectoria}{El caso de uso inicia cuando se va a desplegar la pantalla \IUref{IU35}{Instructores}}
	\UCpaso extrae todos los instructores que no cuentan con la bandera de borrado.
	\UCpaso genera una lista de instructores con los instructores del paso [1].
	\UCpaso muestra la pantalla  \IUref{IU35}{Instructores} con la lista del paso [2].
\end{UCtrayectoria}

%--------------------------------------		
\begin{UCtrayectoriaA}{A}{Todos los instructores se encuentran bloqueados o no hay instructores registrados}
	\UCpaso despliega en la pantalla \IUref{IU35}{Instructores} con el mensaje {\bf MSG13-} ''No hay instructores disponibles''.
\end{UCtrayectoriaA}

%--------------------------------------


% Copie este bloque por cada caso de uso:
%-------------------------------------- COMIENZA descripción del caso de uso.

%\begin{UseCase}[archivo de imágen]{UCX}{Nombre del Caso de uso}{
%--------------------------------------
	\begin{UseCase}{CU36}{Modificar Instructor}{
		Poder modificar los datos personales de un instructor y sus actividades asignadas.
	}
		\UCitem{Versión}{\color{Gray}0.4}
		\UCitem{Autor}{\color{Gray}Quiroz Olmedo Luis Eduardo}
		\UCitem{Supervisa}{\color{Gray}Francisco}
		\UCitem{Actor}{Gerente}
		\UCitem{Propósito}{Que los datos ingresados al momento de dar de alta un instructor puedan modificarse, así como también poder asignar nuevas actividades o editar las existentes asignadas a un instructor.}
		\UCitem{Entradas}{Instructor}
		\UCitem{Origen}{Teclado}
		\UCitem{Salidas}{Mensaje de confirmación}
		\UCitem{Destino}{Pantalla de Instructores.}
		\UCitem{Precondiciones}{ •Que el actor tenga una sesión iniciada en el sistema.\newline
        • Que el sistema tenga al menos un instructor y una actividad registrada.
}
		\UCitem{Postcondiciones}{El Instructor seleccionado ha sido modificado y se ve el cambio en la pantalla de Intructores}
		\UCitem{Errores}{
		 • El actor dejo los campos Nombre y Dirección completamente vacíos.\textbf{Trayectoria Alternativa B}\newline
		 El actor dejo el campo de “Actividades Asignadas” sin ninguna actividad.\textbf{Trayectoria Alternativa C}.
		}
		\UCitem{Tipo}{Caso de uso primario}
		\UCitem{Observaciones}{}
	\end{UseCase}
%--------------------------------------
	\begin{UCtrayectoria}{Este caso de uso inicia cuando un actor se encuentra en la pantalla \IUref{IU35}{Pantalla de Instructores} y tiene una sesión iniciada.}
		\UCpaso[\UCactor] selecciona el icono de edición sobre el instructor que se va a modificar.
		\UCpaso despliega la pantalla \IUref{IU352}{Pantalla Editar Instructor}
		\UCpaso permite editar los campos de texto “Nombre” y “Dirección”.
		\UCpaso[\UCactor] introduce un Nombre y Dirección nuevos.
		\UCpaso consulta las actividades con horarios libres y las despliega en la pantalla en forma de lista.
		\UCpaso[\UCactor] selecciona una actividad.
		\UCpaso consulta los horarios disponibles de la actividad seleccionada y los despliega en la pantalla en forma de lista.
		\UCpaso[\UCactor] selecciona un horario.
		\UCpaso[\UCactor] selecciona el boton guardar.
		\UCpaso actualiza los campos de nombre y dirección con los valores introducidos en el paso 4.
		\UCpaso \IUref {CU38}
		\UCpaso despliega la pantalla \IUref{IU352}{Instructores} con el mensaje {\bf MSG4-}``El Instructor [{\em Nombre de Instructor}] ha sido actualizado con éxito.''.
	\end{UCtrayectoria}

%--------------------------------------		
		\begin{UCtrayectoriaA}{A}{El actor NO introduce un nombre o dirección nuevos.}
			\UCpaso deja los campos “Nombre” y “Dirección” sin ningún cambio.
		\end{UCtrayectoriaA}
		
%--------------------------------------
		\begin{UCtrayectoriaA}{B}{El actor deja vacío el campo “Nombre” o “Dirección” o ambos campos.}
			\UCpaso despliega un mensaje dentro del campo con la leyenda {\bf MSG5-}``Campo Requerido'' .
		\end{UCtrayectoriaA}
%--------------------------------------
		\begin{UCtrayectoriaA}{C}{El actor NO selecciona ninguna actividad.}
			\UCpaso despliega un mensaje en pantalla con la leyenda {\bf MSG6-}``Asignar al menos una actividad'' .
		\end{UCtrayectoriaA}
%--------------------------------------
		\begin{UCtrayectoriaA}{D}{El actor selecciona el icono de borrado de la actividad }
			\UCpaso muestra una alerta en pantalla con el mensaje “Seguro que deseas eliminar la actividad” y las opciones “Aceptar” y “Cancelar”.
		    \UCpaso[\UCactor] selecciona la opción Aceptar.
		    \UCpaso elimina la relacion de con el horario y la actividad y la elimina de la lista de actividades asignadas.
		\end{UCtrayectoriaA}
%--------------------------------------
		\begin{UCtrayectoriaA}{E}{El actor selecciona la pestaña “Nueva Actividad”.}
			\UCpaso ejecuta los pasos [5-12].
		\end{UCtrayectoriaA}
%--------------------------------------
		\begin{UCtrayectoriaA}{F}{El sistema no tiene actividades disponibles}
			\UCpaso Muestra una alerta en pantalla con el mensaje "No hay actividades disponibles".
		\end{UCtrayectoriaA}		
%--------------------------------------
%\begin{UseCase}[archivo de imágen]{UCX}{Nombre del Caso de uso}{
%--------------------------------------
\begin{UseCase}{CU37}{Eliminar Instructor}{
		Eliminación de una entidad Instructor levantando una bandera de borrado sin eliminar los datos del Instructor.
	}
	\UCitem{Versión}{\color{Gray}0.4}
	\UCitem{Autor}{\color{Gray}Quiroz Olmedo Luis}
	\UCitem{Supervisa}{\color{Gray}Francisco}
	\UCitem{Actor}{Gerente}
	\UCitem{Propósito}{Evitar que se le asignen actividades al instructor que se desea borrar pero conservando sus datos personales.}
	\UCitem{Entradas}{Instructor}
	\UCitem{Origen}{Mouse}
	\UCitem{Salidas}{Mensaje de confirmación}
	\UCitem{Destino}{Pantalla de Instructores.}
	\UCitem{Precondiciones}{• El actor debe tener una sesión iniciada\newline
	• Debe exister al menos un instructor registrado}
	\UCitem{Postcondiciones}{El Instructor seleccionado ha sido eliminado.}
	\UCitem{Errores}{-}
	\UCitem{Tipo}{Caso de uso primario}
	\UCitem{Observaciones}{-}
\end{UseCase}
%--------------------------------------
\begin{UCtrayectoria}{Este caso de uso inicia cuando un actor se encuentra en la pantalla \IUref{IU352}{Pantalla Editar Instructor}}
	\UCpaso[\UCactor] selecciona el botón ''Eliminar'' en la columna del instructor que desea eliminar.
	\UCpaso despliega una alerta en pantalla {\bf ALERT5-} ''Seguro que deseas eliminar al instructor'' con las opciones ''Aceptar'' y ''Cancelar''.
	\UCpaso[\UCactor]selecciona la opción ''Aceptar''.
	\UCpaso levanta una bandera de borrado a la entidad del instructor seleccionado.
	\UCpaso desvincula los horarios de las actividades asiganadas al instructor borrado y las deja visibles para su posible asignación a otros instructores.
	\UCpaso borra al instructor de la lista de instructores en la pantalla \IUref{IU352}{Pantalla Editar Instructor}
	\UCpaso despliega la pantalla \IUref{IU352}{Pantalla Editar Instructor} con el mensaje {\bf MSG12-} ''Borrado Exitoso''.
\end{UCtrayectoria}
%--------------------------------------		
\begin{UCtrayectoriaA}{A}{El actor selecciona el botón “Cancelar”}

	\UCpaso despliega la pantalla \IUref{IU352}{Pantalla Editar Instructor} sin ejecutar ningún cambio.

\end{UCtrayectoriaA}


%--------------------------------------

% Copie este bloque por cada caso de uso:
%-------------------------------------- COMIENZA descripción del caso de uso.

%\begin{UseCase}[archivo de imágen]{UCX}{Nombre del Caso de uso}{
%--------------------------------------
\begin{UseCase}{CU38}{Asignar Actividad al Instructor}{
	Asignar una entidad Actividad seleccionada previamente con una entidad Instructor.
	}
	\UCitem{Versión}{\color{Gray}0.4}
	\UCitem{Autor}{\color{Gray}Quiroz Olmedo Luis}
	\UCitem{Supervisa}{\color{Gray}Francisco}
	\UCitem{Actor}{Gerente}
	\UCitem{Propósito}{Crear una unión con una actividad seleccionada y un horario y para evitar duplicados en asignaciones de actividades}
	\UCitem{Entradas}{Actividad}
	\UCitem{Origen}{Mouse y teclado}
	\UCitem{Salidas}{-}
	\UCitem{Destino}{Pantalla de Editar Instructor o Registrar Instructor}
	\UCitem{Precondiciones}{• Que el sistema tenga al menos un instructor y una actividad registrados.\newline
	• Que el actor haya seleccionado una actividad.
	}
	\UCitem{Postcondiciones}{Quedara unido un horario de la Actividad seleccionada con un Instructor.}
	\UCitem{Errores}{Formulario no este llenado de acuerdo al formato.}
	\UCitem{Tipo}{Caso de uso secundario}
	
\end{UseCase}
%--------------------------------------
\begin{UCtrayectoria}{Este caso de uso inicia cuando un actor se encuentra en la pantalla de \IUref{IU35}{Instructor} o en la pantalla \IUref{IU352}{Pantalla Editar Instructor}  y seleccionó una actividad disponible}
	\UCpaso consulta los horarios que no han sido asignados de la actividad seleccionada y los despliega en la pantalla en forma de lista.
	\UCpaso[\UCactor] selecciona el horario.
	\UCpaso el sistema elimina de la lista de horarios disponibles el horario seleccionado.
	\UCpaso el sistema reserva el horario para que ya no sea mostrado en alguna sesion existente.
	\UCpaso el sistema liga el horario seleccionado con el Instructor que se esta editando.
\end{UCtrayectoria}

%\BRref{BR129}{Determinar si un Estudiante puede inscribir Seminario.} \Trayref{A}
%--------------------------------------		
\begin{UCtrayectoriaA}{A}{No hay horarios disponibles de la actividad seleccionada.}
	\UCpaso muestra un mensaje en pantalla con la leyenda {\bf MSG11-} ''Sin horarios disponibles''.
	\UCpaso redirecciona ala pantalla \IUref{IU352}{Pantalla Editar Instructor} o  \IUref{IU35}{Instructor} dependiendo de donde venga la selección.
	%\UCpaso[\UCactor] Continua con el paso 8 del \UCref{CU3}.
\end{UCtrayectoriaA}
%--------------------------------------		
\begin{UCtrayectoriaA}{B}{El actor no selecciona ningún horario}

	\UCpaso muestra un mensaje en pantalla con la leyenda {\bf MSG12-} ''Se debe seleccionar al menos un horario'' y se continua en el paso [2].
\end{UCtrayectoriaA}


% Copie este bloque por cada caso de uso

%-------------------------------------- TERMINA descripción del caso de uso.


\begin{UseCase}{CU39}{Vender membresia a clientes}{
		El actor desde su pagina de inicio da clic al boton de vender membresia en la cual se le pide el correo al cliente y despues se selecciona la membresia que escoge el cliente y despues se genera el contrato para el cliente.}
	\UCitem{Versión}{\color{Gray}0.1}
	\UCitem{Autor}{\color{Gray}Daniel}
	\UCitem{Supervisa}{\color{Gray}Luis}
	\UCitem{Actor}{\hyperlink{Vendedor}{Vendedor}}
	\UCitem{Propósito}{ Que el cliente pueda tener acceso a las instalaciones y que se pueda realizar la venta con mas facilidad.}
	\UCitem{Entradas}{Correo del cliente,nombe del cliente,tipo de membresia.}
	\UCitem{Origen}{Mouse, teclado}
	\UCitem{Salidas}{Datos generales del cliente,Contrato de venta en pdf.}
	\UCitem{Destino}{Pantalla del actor,Impresora.}
	\UCitem{Precondiciones}{ Que el cliente no tenga una membresia del mimso tipo.}
	\UCitem{Postcondiciones}{Se registrara el cliente en el sistema.}
	\UCitem{Errores}{Que el cliente no este regisrado en el sistema.}
	\UCitem{Observaciones}{ninguna.}
\end{UseCase}
%--------------------------------------

\begin{UCtrayectoria}{Principal}
	\UCpaso[\UCactor] Accede la página de ventas de membresia.
	\UCpaso Despliega una pantalla con un campo para ingresar el correo y un boton para realizar la busqueda.
	\UCpaso[\UCactor] Ingresa el correo en el campo.  
	\UCpaso[\UCactor] presiona el botón de buscar.
	\UCpaso Valida que el correo este en el sistema.\Trayref{A}.
	\UCpaso Muestra una pantalla con los datos del cliente y los tipos de membresia que se tiene para poder seleccionar solo una de ellas.
	\UCpaso [\UCactor] Seleciona el tipo de membresia que pidio el cliente.
	\UCpaso Muestra un mensaje de confirmacion de la compra con dos botones: aceptar y cancelar.
	\UCpaso[\UCactor] Seleciona aceptar.\Trayref{B}
	\UCpaso añade en el sistema que el cliente tiene ya una membresia.
	\UCpaso Genera el contrato y lo manda a la impresora.
\end{UCtrayectoria}

%--------------------------------------		
		\begin{UCtrayectoriaA}{A}{El cliente no esta en registrado en el sistema.}
			\UCpaso Muestra mensaje que no esta registrado .
		\end{UCtrayectoriaA}
%--------------------------------------		
		\begin{UCtrayectoriaA}{B}{El cliente no accede a la venta.}
			\UCpaso[\UCactor] Selecciona el boton de cancelar.
			\UCpaso Muestra la pagina principal de venta donde se pide el correo del cliente.
		\end{UCtrayectoriaA}
		
%-------------------------------------- TERMINA descripción del caso de uso.

% Copie este bloque por cada caso de uso:


