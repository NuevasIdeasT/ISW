\section{Guión de Pruebas}

\subsection{Guión de prueba 1}
\begin{itemize}
\item ID: Guión 01
\item ALCANCE: \UCref{CU1}
\item NOMBRE: Iniciar sesión de jefe de inmobiliario  
\item PREPARACIÓN:\\
-En la BD debe de estar registrado el jefe de inmobiliario.
\item CP 1:
\item DATOS DE ENTRADA:\\
	-Correo electrónico del J.I: Jefe\_inmobiliario$@$hotmail.com\\
	-Contraseña: SOY\_elJefe1
\begin{center}			
	\begin{tabular}{|l|l|l|l|}
		\hline
		CP 1\\
		PASOS\\
		\hline 1.- Introducir el correo electrónico\\
		\hline 2.- Introducir la contraseña.\\
		\hline 3.- Presionar el botón Iniciar Sesión\\
		\hline
	\end{tabular}
\end{center}
\item SALIDAS: \IUref{IU1}{Pantalla de Inicio de Sesión del Jefe de Inmobiliario.}\label{CU1LoginJI}
\item CP 2:
\item DATOS DE ENTRADA:\\
	-Correo electrónico: cualquier\_cosa$@$hotmail.com\\
	-contraseña: caulquiera
\begin{center}			
	\begin{tabular}{|l|l|l|l|}
		\hline
		CP 2\\
		PASOS\\
		\hline 1.- Introducir el correo electrónico\\
		\hline 2.- Introducir la contraseña.\\
		\hline 3.- Presionar el botón Iniciar Sesión\\
		\hline
	\end{tabular}
\end{center}
\item SALIDAS: Mensaje {\bf MSG1-}``El empleado [{\em Nombre de sesión}] no existe.''.
  
\item CP 3:
\item DATOS DE ENTRADA:\\
	-Correo electrónico del J.I: Jefe\_inmobiliario$@$hotmail.com
\begin{center}			
	\begin{tabular}{|l|l|l|l|}
		\hline
		CP 3\\
		PASOS\\
		\hline 1.- Introducir el correo electrónico\\
		\hline 2.- No introducir la contraseña.\\
		\hline 3.- Presionar el botón Iniciar Sesión\\
		\hline
	\end{tabular}
\end{center}
\item SALIDAS:  Mensaje {\bf MSG3-}``Todos los campos son obligatorios.''.
\item CP 4:
\item DATOS DE ENTRADA:\\
-Correo electrónico: jefe\_inmobiliario$@$hotmail.com\\
-contraseña: caulquiera
\begin{center}			
	\begin{tabular}{|l|l|l|l|}
		\hline
		CP 2\\
		PASOS\\
		\hline 1.- Introducir el correo electrónico\\
		\hline 2.- Introducir la contraseña.\\
		\hline 3.- Presionar el botón Iniciar Sesión\\
		\hline
	\end{tabular}
\end{center}
\item SALIDAS: Mensaje {\bf MSG12-}``El password no es correcto.''.
\end{itemize}

%%%para otro caso de uso%%%%%%
\subsection{Guión de prueba 2}
\begin{itemize}
	\item ID: Guión 02
	\item ALCANCE: \UCref{CU2}
	\item NOMBRE: Consultar Actividades de sucursal  
	\item PREPARACIÓN:\\
	-En la BD debe de estar registrado el jefe de inmobiliario.\\
	-En la BD debe de estar registrado al menos una actividad.\\
	-En la BD debe de estar registrado al menos una sucursal.\\
	\item CP 1:
	\item DATOS DE ENTRADA:\\
	-
	\begin{center}			
		\begin{tabular}{|l|l|l|l|}
			\hline
			CP 1\\
			PASOS\\
			\hline 1.- Inicia sesión del jefe de inmobiliario.\\
			\hline 2.- Precionar el botón de actividades.\\
			\hline
		\end{tabular}
	\end{center}
	\item SALIDAS: \IUref{IU2}{ Pantalla de Actividades.}\label{CU1LoginJI}
	
\end{itemize}

%%%para otro caso de uso%%%%%%
\subsection{Guión de prueba 3}
\begin{itemize}
	\item ID: Guión 03
	\item ALCANCE: \UCref{CU3}
	\item NOMBRE: Modificar actividad de sucursal 
	\item PREPARACIÓN:\\
	-En la BD debe de estar registrado el jefe de inmobiliario.\\
	-En la BD debe de estar registrado al menos una actividad.\\
	-En la BD debe de estar registrado al menos una sucursal.\\
	-En la BD debe de estar registrado el instructor con el que se harán las pruebas.\\
	-En la BD debe estar llenos todos los campos de esa pagina.\\
	\item CP 1:
	\item DATOS DE ENTRADA:\\
	-Nombre de Actividad: Natación\\
	-hora(s): lunes-13:00, miercoles-13:00, jueves-13:00, viernes-13:00, domingo-13:00\\
	-Instructor: Juanito perez\\
	-Fotografia: foto png\\
	-Descripción: Natación es una disciplina...
	\begin{center}			
		\begin{tabular}{|l|l|l|l|}
			\hline
			CP 1\\
			PASOS\\
			\hline 1.- Inicia sesión del jefe de inmobiliario.\\
			\hline 2.- Precionar el botón de actividades.\\
			\hline 3.- Precionar el botón de editar.\\
			\hline 4.- Borrar los datos que se van a modificar.\\
			\hline 5.- Escribir el  nombre de la actividad.\\
			\hline 6.- Escribir en la casilla correspondiente la hora en la tabla de dias de la semana.\\
			\hline 7.- Escribir la descripción.\\
			\hline 8.- Presionar el botón de aceptar.\\
			\hline
		\end{tabular}
	\end{center}
	\item SALIDAS: Mensaje {\bf MSG5-}``Operación éxitosa.''.
	
	\item CP 2:
	\item DATOS DE ENTRADA:\\
	-Nombre de Actividad: Natación\\
	-hora(s): lunes-una hrs, miercoles-una hrs, jueves-una hrs, viernes-una hrs, domingo-una hrs\\
	-Instructor: Juanito perez\\
	-Fotografia: foto png\\
	-Descripción: Natación es una disciplina...
	\begin{center}			
		\begin{tabular}{|l|l|l|l|}
			\hline
			CP 1\\
			PASOS\\
			\hline 1.- Inicia sesión del jefe de inmobiliario.\\
			\hline 2.- Precionar el botón de actividades.\\
			\hline 3.- Precionar el botón de editar.\\
			\hline 4.- Borrar los datos que se van a modificar.\\
			\hline 5.- Escribir el  nombre de la actividad.\\
			\hline 6.- Escribir en la casilla correspondiente la hora en la tabla
			de dias de la semana.\\
			\hline 7.- Escribir la descripción.\\
			\hline 8.- Presionar el botón de aceptar.\\
			\hline
		\end{tabular}
	\end{center}
	\item SALIDAS: Mensaje {\bf MSG5-}``La fecha debe ser 0:00-24:00.''.
	
\end{itemize}
