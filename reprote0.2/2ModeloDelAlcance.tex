%=========================================================
\chapter{Modelo del Alcance}
\label{cap:reqUsr}

	En este capítulo se modela el alcance del sistema. Se presentan inicialmente los Actores involucrados y sus requerimientos. Después se presentan los requerimientos funcionales se presenta el modelo Físico y Lógico del sistema.


%---------------------------------------------------------
\section{Modelado de Usuarios}
%\cdtInstrucciones{
%	Identifique los actores que estarán involucrados en los procesos relacionados con el sistema para esta iteración de desarrollo. Ponga énfasis en los procesos involucrados.
%}

\subsection{Organigrama de la Empresa}



\begin{figure}[htbp]
	\begin{center}
		\includegraphics[width=.8\textwidth]{images/organigrama}
		\caption{Organigrama de la empresa}
		\label{fig:organigrama}
	\end{center}
\end{figure}

%=========COPIA DESDE AQUI PARA AGREGAR A TU ACTOR=========
%---------------------------------------------------------
\begin{Usuario}{\subsection{Jefe de Inmobiliario}}{
		Es el encargado de todas las operaciones que tengan que ver con con la sucursal, como lo son áreas, actividades y propiamente las sucursales. Mencionando que es el único que puede modificar las operaciones de las sucursales.
	}
		\item[Responsabilidades:] \cdtEmpty
	\begin{itemize}
		\item Supervisar la información de actividades, áreas y sucursales.
		\item Agrega o modifica los datos.
	\end{itemize}
	
	\item[Perfil:] \cdtEmpty
	\begin{itemize}
		\item Amplia experiencia en el ramo.
		\item Licenciatura como mínimo.
	\end{itemize}
	\item[Procesos en los que participa:] \cdtEmpty
	\begin{itemize}
		\item PC-V01 Aprobar la agregación de nuevas sucursales.
		
	\end{itemize}
\end{Usuario}

%---------------------------------------------------------
%=======HASTA AQUI TERMINA PARA AGREGAR A TU ACTOR========

%=========COPIA DESDE AQUI PARA AGREGAR A TU ACTOR=========
%---------------------------------------------------------
\begin{Usuario}{\subsection{Gerente}}{
		Es el encargado de la sucursal en la que trabaja, responsable de realizar modificaciones de actividades, áreas, trabajadores y todo lo que tenga que ver con la propia sucursal.
	}
	\item[Responsabilidades:] \cdtEmpty
	\begin{itemize}
		\item Supervisar que los procesos se realicen correctamente.
		\item Agrega o modifica los datos.
	\end{itemize}
	
	\item[Perfil:] \cdtEmpty
	\begin{itemize}
		\item Amplia experiencia en el ramo.
		\item Licenciatura como mínimo.
	\end{itemize}
	\item[Procesos en los que participa:] \cdtEmpty
	\begin{itemize}
		\item PC-V01 Aprobar la agregación de nuevas actividades.
		\item PC-V02 Supervisar que los demás procesos se cumplan
		\item PC-V03 Elaborar informe incidencias en la sucursal.
	\end{itemize}
\end{Usuario}

%---------------------------------------------------------
%=======HASTA AQUI TERMINA PARA AGREGAR A TU ACTOR========

%=========COPIA DESDE AQUI PARA AGREGAR A TU ACTOR=========
%---------------------------------------------------------
\begin{Usuario}{\subsection{Ventas}}{
		Es el encargado de la venta de membresías y de registrar a los clientes cuando la adquieran y registrar a los invitados.
	}
	\item[Responsabilidades:] \cdtEmpty
	\begin{itemize}
		\item Genera citas para los clientes cuando entren como invitados.
		\item Agrega datos de los clientes.
		\item 
	\end{itemize}
	
	\item[Perfil:] \cdtEmpty
	\begin{itemize}
		\item Amplia experiencia en el ramo.
		\item Licenciatura como mínimo.
	\end{itemize}
	
\end{Usuario}

%---------------------------------------------------------
%=======HASTA AQUI TERMINA PARA AGREGAR A TU ACTOR========

%=========COPIA DESDE AQUI PARA AGREGAR A TU ACTOR=========
%---------------------------------------------------------
\begin{Usuario}{\subsection{Cliente}}{
		No es encargado de ninguna parte de la sucursal, mas que asistir a la sucursal si es que paga su membresía.
	}
			\item[Responsabilidades:] \cdtEmpty
		\begin{itemize}
			\item hacer buen uso de las instalaciones
		\end{itemize}
		
		\item[Perfil:] \cdtEmpty
		\begin{itemize}
			\item Cualquier perfil
		\end{itemize}
		\item[Procesos en los que participa:] \cdtEmpty
		\begin{itemize}
			\item PC-V01 Ingreso a la sucursal.
			
		\end{itemize}

\end{Usuario}
%---------------------------------------------------------
%=======HASTA AQUI TERMINA PARA AGREGAR A TU ACTOR========


%---------------------------------------------------------
\section{Requerimientos de usuario}

%\cdtInstrucciones{
%	Identifique y describa los requerimientos funcionales del sistema señalando: id, nombre, descripción y prioridad.
%}

\begin{table}[htbp!]
	\begin{requerimientosU}
		\FRitem{RU1}{Registro de clientes}{El usuario requiere llevar un registro actualizado de todos los clientes para su seguimiento, atención teniendo por datos el nombre completo, domicilio, correo, curp, teléfono, número de emergencia, fecha de nacimiento y datos medicos como estatura, peso, alergias y enfermedades crónicas, si toma algún medicamento.}{1}{\PLAN}
		
		\FRitem{RU2}{Registro de áreas}{El usuario requiere llevar un registro actualizado de todas las áreas que tenga la sucursal teniendo como datos la capcidad máxima de personas, encargado del área, actividades impartidas, la(s) sucursal(es) donde se encuentra y horario(s).}{1}{\PLAN}
		
		\FRitem{RU3}{Registro de actividades}{El usuario requiere llevar un registro actualizado de todas las actividades que tenga la sucursal teniendo como datos nombre de a actividad, dia y hora en que se imparte, que instructor tiene, cuantos estan inscritos, etc.}{1}{\PLAN}
		
		\FRitem{RU4}{Registro de sucursal}{El usuario requiere llevar un registro actualizado de todas las sucursales que haya en la franquicia.}{1}{\PLAN}
		
		\FRitem{RU5}{Planeación de personal}{El usuario requiere llevar un registro de los roles que tiene cada empleado de la sucursal.}{1}{\PLAN}
		\FRitem{RU6}{Visualización de infromación}{El usuario requiere ver la información almacenada la información desplegando pantallas.}{1}{\PLAN}
		\FRitem{RU7}{Registro de pago}{El usuario requiere ver la información de los clientes que realizan su pago.}{1}{\PLAN}
		\FRitem{RU8}{Aviso de pago}{El usuario requiere informar al cliente de su próximo pago.}{1}{\PLAN}
		\FRitem{RU9}{Registro de ingreso}{El usuario requiere ver la información de quién ingresa a la sucursal.}{1}{\PLAN}
		\FRitem{RU10}{Busqueda Actividades populares}{El usuario requiere ver cuales son las actividades que mas clientes adquieren.}{1}{\PLAN}
		\FRitem{RU11}{Conteo de membresías}{El usuario requiere saber el tipo de membresía que mas vende.}{1}{\PLAN}
		\FRitem{RU12}{Identificación de nichos de mercado}{El usuario requiere ver cuales son sus clientes principales, divididos por genero, edad, zona geográfica.}{1}{\PLAN}
		\FRitem{RU9}{Retencion de clientes}{El usuario requiere mejoras de servicio para retener a los clientes registrados.}{1}{\PLAN}
		
	\end{requerimientosU}
	\caption{Requerimientos funcionales del sistema.}
	{\footnotesize\em Para leer correctamente esta tabla vea la leyenda en la Tabla~\ref{tbl:leyendaRF} en la página~\pageref{tbl:leyendaRF}.}
	\label{tbl:reqFunc}
\end{table}



%---------------------------------------------------------
\section{Especificación de plataforma}	

\cdtInstrucciones{
%	Coloque un diagrama y su descripción para aclarar el tipo de solución propuesta. \\
	
	En esta sección se debe aclarar:
	
	\begin{description}
		\item[Tipo de sistema:] Sistema web.
		\item[Software requerido:] S.O. de 64 bits, preferible linux, phyton 2.7 V. Django 1.9, Framework bootstrap.
		\item[Hardware requerido:] Computadora con arquitectura de 64 bits, procesador para arquitectura mencionada, RAM 4 GB (recomendado).
		\item[servicios:] .
	\end{description}
}

\begin{figure}[htbp!]
	\begin{center}
		\fbox{\includegraphics[width=.6\textwidth]{images/arquitectura}}
		\caption{Arquitectura del sistema.}
		\label{fig:arquitectura}
	\end{center}
\end{figure}

En la figura~\ref{fig:arquitectura} se describe la estructura del sistema, en ella se detalla ...